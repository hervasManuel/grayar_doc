\chapter{Listado de acrónimos}

{\small
\begin{acronym}[XXXXXXXX]
  \Acro{GNU}     {\acs{GNU} is Not Unix}
  \acro{OO}      {Orientación a Objetos}
  \Acro{CLI}     {Command Line Interface}
  \Acro{GUI}     {Graphical User Interface}
  \Acro{HTML}    {HyperText Markup Language}
  \Acro{API}     {Application Programming Interface}
  \Acro{SDK}     {Software Development Kit}
  \Acro{RGB}     {Red - Green - Blue}
  \Acro{GPS}     {Global Position System}
  \Acro{HDM}     {Head Mount Display}
  \Acro{ARM}     {Advanced RISC Machine}
  \Acro{SBC}     {Single Board Computer}
  \Acro{RAM}     {Random Access Memory}
  \Acro{USB}     {Universal Serial Bus}
  \Acro{OCR}     {Optical Character Recognition}
  \Acro{FPS}     {Frames per Second}
  \Acro{IR}      {Infrared}
  \Acro{XML}     {eXtensible Markup Language}
  \Acro{KISS}    {Keep It Simple, Stupid!}
  \Acro{ARCO}    {Arquitectura y Redes de COmputadores}

\end{acronym}
}


% \ac{OO}   la primera vez \acf, después \acs
% \acs{OO}  short: OO
% \acf{OO}  full : Object Oriented (OO)
% \acl{OO}  large: Object Oriented
% \acx{OO}         OO (Object Oriented)

% usa \Acro cuando no debe aparecer nunca expandido en el texto

% Local variables:
%   TeX-master: "main.tex"
% End:
