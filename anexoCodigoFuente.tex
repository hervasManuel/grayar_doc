\chapter{Código fuente}
\label{chap:anexo_codigo_fuente}

Dada la extensión del código fuente, éste no ha sido incluido en la documentación. Se incluye por
tanto de forma digital en el CD adjunto a este documento.

\section{\textit{ARgos} Cliente}
Para la aplicación cliente, es decir, aquella que se ejecuta en la Raspberry Pi, proyecto presenta
la siguiente estructura de directorios

\begin{itemize}
\item \textbf{include:} Contiene los archivo de cabecera o \textit{headers} C++ del proyecto.
\item \textbf{libs:} Contiene las biblioteca externas \texttt{freetypeGlesRpi} y \texttt{glm}.
\item \textbf{media:} Contiene los recursos del sistema categorizados en subdirectorios:
  \begin{itemize}
    \item \textbf{fonts:} Contiene las fuentes \textit{true-type} a usar por el sistema.
    \item \textbf{images:} Contiene todas las imágenes del sistema.
    \item \textbf{sounds:} Contiene todos los archivos de audio del sistema.
  \end{itemize}
\item \textbf{shaders:} Contiene todos los programas \textit{vertex shader} y \textit{fragment
    shader} usados por OpenGL ES 2.0.
\item \textbf{src:} Contiene los archivos de código fuente C++ del proyecto.
\item El directorio raíz contiene los archivos de calibración de la cámara y el \textit{Makefile}
  necesario para la construcción del proyecto.
\end{itemize}

\section{\textit{ARgos} Servidor}
Por otra parte, la aplicación servidora o aquella que se ejecuta en el computador externo presenta
la siguiente estructura de directorios

\begin{itemize}
\item \textbf{data:} Contiene los archivos de configuración \acs{XML} y \textit{scripts} del sistema.
\item \textbf{include:} Contiene los archivo de cabecera o \textit{headers} C++ del proyecto.
\item \textbf{src:} Contiene los archivos de código fuente C++ del proyecto.
\item El directorio raíz contiene el \textit{Makefile} necesario para la construcción del proyecto y
  el archivo de descriptores empleado para el reconocimiento de documentos.
\end{itemize}

\section{\textit{CalibrationToolbox} Servidor}
\begin{itemize}
\item \textbf{data:} Contiene los archivos de calibrado YAML. 
\item \textbf{include:} Contiene los archivo de cabecera o \textit{headers} C++ la aplicación.
\item \textbf{src:} Contiene los archivos de código fuente C++ de la aplicación.
\end{itemize}

\section{\textit{CalibrationToolbox} Cliente}
\begin{itemize}
\item \textbf{include:} Contiene los archivo de cabecera o \textit{headers} C++ de la aplicación.
\item \textbf{libs:} Contiene las biblioteca externas \texttt{freetypeGlesRpi} y \texttt{glm}.
\item \textbf{src:} Contiene los archivos de código fuente C++ de la aplicación.
\end{itemize}


% Local Variables:
% TeX-master: "main.tex"
%  coding: utf-8
%  mode: latex
%  mode: flyspell
%  ispell-local-dictionary: "castellano8"
% End:
