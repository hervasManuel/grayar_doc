\chapter{Manual de \textit{calibrationToolBox}}

El proceso de calibrado del sistema se ha creado como una utilidad independiente del sistema principal de \textit{ARgos}. Una vez calibrado, la aplicación proporciona los ficheros YAML, con los parámetros intrínsecos y extrínsecos, que necesita \textit{ARgos} para el calculo del registro.

El proceso está separado en una aplicación cliente \textit{(calibrationToolBox\_client)} que está alojada en la Raspberry Pi, y otra aplicación servidor \textit{(calibrationToolBox\_server)} que se podría ejecutar en cualquier otro equipo.

En un principio se diseñó como un único programa que ejecutaba en la Raspberry Pi, pero si se tomaban muchas capturas para conseguir una calibración más precisa, el sistema se ralentizaba, y el tiempo de calibrado crecía excesivamente.  

Se ha empleado esencialmente el enfoque de Zhang~\cite{Zhang} para calibrado de cámaras. Se utiliza un patrón tipo tablero de ajedrez, en la que se alternan cuadrados blancos y negros, de dimensiones conocidas. El patrón se imprime y se pega sobre una superficie plana rígida. Se debe dejar una zona despejada a continuación del tablero de ajedrez, ya que en esta zona se proyectará el patrón que utiliza el proyector para su calibrado. Se debe prestar atención en que orientación es pegado, ya que si se colocase al revés, la proyección dinámica se realizaría fuera del panel.

Antes de iniciar el proceso de calibrado, debemos definir los parámetros de configuración;

\begin{itemize} 
\item Tomamos la medida del lado de un cuadrado para proporcionarla al programa de calibrado, de esta forma las unidades de medida del sistema de referencia se corresponden a medidas reales. 

\item Además del tamaño del lado debemos indicar el número de esquinas en vertical y horizontal que tiene el patrón que estamos utilizando. Gracias a esto, no estamos atados a realizar el calibrado con un patrón particular, que tenga unas medidas concretas. Basta indicar las características del patrón que vamos a utilizar para realizar un calibrado correcto.

\item Los otros dos parámetros de la configuración, corresponden al desplazamiento (en los ejes $X$ e $Y$) de la proyección del patrón de círculos asimétrico. El objetivo es ajustar la visualización de la proyección para que el patrón proyectado aparezca junto al patrón impreso. 

\item Los últimos parámetros son las dimensiones de las imágenes, tanto de la cámara como del proyector. Se pueden utilizar resoluciones distintas, pero en GrayAR, se he optado por utilizar la misma en ambos dispositivos.  
\end{itemize}

A continuación, colocamos la superficie bajo la cámara, en distintas orientaciones y distancias, para obtener una serie de imágenes en los que se encuentre visible el patrón.

Cuando el proceso disponga de al menos 20 imágenes válidas, realizará los cálculos para obtener los parámetros intrínsecos de cámara y los grabará en el fichero \textbf{calibrationCamera.yml}

Se considera una captura válida aquella que al aplicar los cálculos de calibrado, devuelve un error de reproyección menor de 0.25. 

El siguiente paso consiste en calibrar el proyector. Al comienzo de este estado, se proyecta un patrón de círculos asimétrico, en una posición fija.

Colocar la superficie, de tal forma, que el patrón proyectado quede junto al del tablero de ajedrez. Volver a mover el patrón el espacio de la  proyección hasta que el sistema adquiera 5 imágenes más. Para la aceptación de estas 5 imágenes, el error de reproyección cometido no debe ser mayor de 0.35.

En ese momento, la calibración pasa al modo dinámico, y los puntos proyectados se generan en función de la posición que se encuentre el tablero de ajedrez.

Tras capturar 15 imágenes con los puntos dinámicos, se refinan los cálculos correspondientes a los parámetros intrínsecos del proyector, y finalmente, se calculan los parámetros extrínsecos (matrices de rotación y traslación) entre la cámara y el proyector.

Los parámetros de calibrado se almacenan en los ficheros \textbf{calibrationProjector.yml} y \textbf{cameraProjectorExtrinsics.yml}.

Una vez realizada la calibración se iniciará un test para comprobar que se ha hecho correctamente. El test consiste en proyectar un círculo sobre las cuatro esquinas exteriores del patrón de tablero de ajedrez. Al mover el patrón los puntos proyectados deben siempre permanecer alineados con las esquinas.

Tanto el número de imágenes que se deben detectar antes de pasar a la siguiente fase, como los errores de reproyección máximos permitidos por la cámara y el proyector, son también configurables. Tras realizar numerosas pruebas variando estos parámetros, los mejores resultados manteniendo el compromiso \textit{precisión-tiempo de calibrado-consumo de recursos}, se llega a la conclusión que los valores antes descritos son los que mejor se ajustan. 

Tal y como se describe, el proceso completo de calibrado se puede realizar en varios minutos. Asimismo, mientras que la cámara y el proyector mantengan su posición y rotación entre ellos, no es necesario realizar una nueva calibración y es posible desplazar todo el sistema.
