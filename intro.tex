\chapter{Introducción}

\drop{L}{a} gestión documental es una actividad que la humanidad lleva desarrollando desde hace muchos siglos. Nació prácticamente al mismo tiempo que la escritura, ante la necesidad de dar fe y registrar por escrito acuerdos comerciales o actos administrativos. A lo largo de la historia el volumen de los fondos documentales ha ido en aumento, sobre todo en el último siglo, lo que ha hecho que la tarea archivística se haya vuelto cada vez más compleja.

Según Robergé~\cite{roberge}, la gestión documental se puede definir como «\textit{el conjunto de las operaciones y de las técnicas relativas a la concepción, el desarrollo, la implantación y la evaluación de los sistemas administrativos necesarios, desde la creación de los documentos hasta su destrucción o su transferencia al archivo permanente cuyo objetivo es lograr eficiencia y economía administrativas}». Por tanto, es evidente suponer que realizando una mejora de la eficiencia en la gestión documental de una organización, vamos a obtener un aumento de la competitividad corporativa.

% La Norma ISO-15489 Internacional Standard on Records Management~\cite{ISO15489-1} aprobada en el año 2001, es el la respuesta a la normalización y uso de buenas prácticas en el proceso de organización y gestión de documentos administrativos  y  su conservación en diferentes soportes y formatos.

%\subsection{Beneficios de la gestión documental}
Entre los principales beneficios, una vez implementado en una organización un programa de gestión de documentos, se pueden encontrar \cite{zapata}:
\begin{itemize}
\item \textbf{Reducción del volumen documental:} Uno de los primeros resultados que se observan, es la disminución de la cantidad de documentos generados, sobre todo de aquellos documentos que no aportan valor informativo. 
\item \textbf{Mejoras en la eficiencia administrativa:} Una vez revisados los procesos de la organización, nos permitirá eliminar aquellos que sean redundantes, innecesarios o muy complejos, cuyo desarrollo implique la realización de tareas sin un objetivo dentro del mismo y que finalmente se derive en el aumento del tiempo de realización del proceso. 
\item \textbf{Incremento de la productividad:} %El beneficio más importante se refleja en el aumento de la productividad de los empleados.
Al producirse una reducción de la documentación, procesos como la  generación, consulta o archivo se simplifican, permitiendo que se dedique más tiempo a las labores propias del negocio.
\item \textbf{Reducción de los costos de infraestructura:} Con la estandarización se obtiene una reducción igualmente de  la infraestructura necesaria para la generación,  conservación y distribución de los documentos,  necesitándose menos equipación, espacio de almacenaje, mobiliario, etc. 
\item \textbf{Aprovechamiento de las tecnologías de la información:} La aplicación de los nuevos conceptos de gestión documental fomentan el uso de las tecnologías de la información, aplicaciones ofimáticas, automatización de tareas, etc. 
\item \textbf{Mejora de los niveles de rentabilidad de la empresa:} La reducción en costos operativos que supone la implementación de un sistema de gestión documental, permite que se pueda invertir en áreas claves del negocio.
\end{itemize}
\bigskip
En la actualidad existe un gran abanico de herramientas software con el objetivo de controlar e incrementar la eficiencia del flujo de documentos que soportan los negocios o sus actividades denominados Sistemas de Gestión Documental (SOD) o, en ingles, Document Management Systems (DMS)

\section{Realidad Aumentada}
La realidad aumentada es una tecnología mediante la cual la percepción del mundo real se complementa con información adicional generada por ordenador en tiempo real. Esta información adicional puede ser desde etiquetas virtuales, representaciones de modelos tridimensionales, o incluso cambios de iluminación. 

%El término Realidad Aumentada fue acuñado en 1990 por Tom Caudell durante el desarrollo de un sistema de Boeing para ayudar a los trabajadores en el ensamblaje de aviones mediante la ayuda de pantallas que mostraban información del montaje. 

La realidad aumentada se puede aplicar a prácticamente todos los campos. En el sector industrial, para la reparación y mantenimiento de máquinas e instalaciones complejas, visualización de datos o simulación.  En aplicaciones médicas, mostrarían la situación de órganos no visibles durante una cirugía. También se han realizado aplicaciones para diseño de interiores (IKEA), presentaciones de productos, educación, publicidad, turismo, arte y ocio. 

%Uno de los principales problemas a resolver es denominado registro. Los objetos del mundo real y virtual deben estar correctamente alineados o la sensación de integración se verá seriamente afectada.

Las aplicaciones de Realidad Aumentada deben cumplir las siguientes características definidas por Ronald T. Azuma \cite{Azuma}:

\begin{description}
\item[Combinar el mundo real con el virtual.] El resultado final debe mostrar la información sintética sobre las imágenes percibidas del mundo real.
\item[Debe ser interactivo en tiempo real.] La integración debe ser realizada \emph{en el momento}, por lo que el cálculo necesario debe realizarse en el menor tiempo posible.
\item[La alineación de los elementos virtuales debe realizarse en 3D.] Los objetos sintéticos deben de estar correctamente alineados en el espacio tridimensional,  o la sensación de integración se verá seriamente afectada.
\end{description}
 
%\subsection{Visión por computador y realidad aumentada}

%La visión por computador es un área relacionada con la  inteligencia artificial que tiene como objetivo la extracción de información del mundo físico a partir de imágenes utilizando sistemas computacionales. 

%La visión artificial, en un intento de imitar o reproducir la capacidad que utiliza el ser humano para interpretar su entorno mediante el sentido de la vista, define tradicionalmente cuatro fases principales ~\cite{velez}: 

%\begin{itemize}

%\item \textbf{Captura:} Consiste en la adquisición de las imágenes mediante algún sensor o cámara. 

%\item \textbf{Procesamiento:} Se realiza un tratamiento digital de las imágenes mediante filtros y transformaciones para eliminar o realzar partes de la imagen sobre las que se realizará el análisis posterior. 

%\item \textbf{Segmentación:} En esta fase se aíslan los elementos que interesan de una escena para comprenderla. 

%\item \textbf{Clasificación:} En ella se intentará distinguir los objetos segmentados realizando un análisis de ciertas características que se definen para diferenciarlos.

%\end{itemize}
%\bigskip
%En la actualidad, las técnicas de visión por computador se utilizan en multitud de entornos industriales. El reconocimiento facial, seguimiento de objetos en tareas de vídeo-vigilancia, búsqueda de imágenes a través de características visuales o el mapeo de una escena para generar un modelo 3D son algunos ejemplos de exitosa implantación comercial.

%La realidad aumentada se podría considerar como la aplicación de distintas técnicas de visión por computador. El principal problema que deben tratar los sistemas de realidad aumentada se considera \textbf{registro}, que consiste en calcular la posición relativa de la cámara respecto a la escena para colocar correctamente las imágenes sintéticas dentro de la  imagen real. Este registro se suele realizar empleando diferentes técnicas, como por ejemplo tracking visual~\cite{gonzalez}.

\section{Reconocimiento y análisis de documentos}

\subsection{Texto y documentos}
En general, podemos considerar que cualquier escena o imagen que tenga un contenido textual como si fuera un documento. Esto incluiría tanto un libro, la matricula de un vehículo o un cartel en una pared. La mayoría de trabajos mediante cámaras están basados en la extracción de texto en imágenes fijas o secuencias de vídeo en las que los autores denominan imágenes naturales, en lugar de imágenes donde el texto está estructurado como los documentos. Ambos enfoques tienen sus desafíos y distintos modos de acometerlos, pero el objetivo final de todos es la de dotar a las cámaras la capacidad de lectura.

En el caso de documentos estructurados, que es el del dominio de este trabajo, las imágenes suelen ser documentos impresos como artículos, cartas, formularios o páginas de libros, donde gran parte de la imagen se asume que es texto, pero también puede contener figuras, diagramas e incluso algunos autores han tratado con anotaciones escritas a mano \cite{Chen}

\subsection{Identificación y recuperación de documentos}
Aunque hoy en día la mayor parte de la producción de información en forma de documentos se realiza por medio de herramientas informáticas (procesadores de texto, correo electrónico, etc.), puede ocurrir, y de hecho será un caso muy habitual, que la información no se restrinja a documentos actuales, ya automatizados, sino que se encuentre impresa. Incluso puede ocurrir que nos interese disponer sólo de la información antigua (archivos y manuscritos).

En estos casos para conseguir una gestión eficaz y ágil, es necesario digitalizar previamente estos documentos para incorporarlos al sistema que tenga implementado la organización.

Las primeras aplicaciones se basaban en el paradigma de reconocimiento de caracteres (OCR), donde se utilizaban estas técnicas para realizar un análisis del contenido informativo de los documentos y utilizarlo para su clasificación y almacenamiento. 

La recuperación de objetos (también nombrada por otros autores reconocimiento o identificación) se incorpora recientemente en la detección de tal manera que un objeto es capturado en una imagen, recuperado de una base de datos y su pose inicial se calcula simultáneamente \cite{Pilet}

El desarrollo de la investigación realizada en este ámbito se inició con los métodos que utilizaban marcas especiales en el documento, como códigos de barras \cite{Graham} o glifos para vincular contenido electrónico con las imágenes capturadas \cite{Hecht}. Los inconvenientes de estos enfoques es que es necesario modificar el formato y la apariencia del documento para introducir las marcas, que en algunos casos pueden distraer al usuario del contenido del documento. Por otro lado, un documento válido para el sistema al que no se le hayan incluido previamente estas marcas, no será detectado y vinculado con la información a recuperar.

La utilización del teléfono móvil y otros dispositivos portátiles para la identificación de documentos ha  que publiquen diversos artículos con algoritmos y métodos que parten de las propias limitaciones de estos dispositivos como es la baja capacidad de computo, la calidad de las imágenes capturadas, en muchos casos borrosas, y la captura parcial del documento.

\section{Limitaciones de la realidad aumentada en la identificación de documentos}
\subsection{Adquisición de imágenes mediante cámaras. Ventajas e inconvenientes}
El análisis de documentos mediante cámaras tiene una serie de ventajas sobre aquellos que están basados en la adquisición mediante escáner. Las cámaras son pequeñas y fáciles de transportar. También se pueden utilizar en cualquier entorno y sobre documentos que por su formato sean difíciles de manipular en un escáner como periódicos, libros, o manuscritos antiguos. Incluso para capturar texto que no se encuentra en papel, como carteles en fachadas, o texto que se encuentre el objetos que se muevan por la escena.

En la mayoría de casos, los escáneres obtienen mejor calidad en la captura de calidad que las realizadas mediante cámaras, pero los sistemas basados en cámaras son mucho más flexibles y portables. 

\subsection{Problemática asociada}
Casi todos los algoritmos de reconocimiento de documentos obtienen grandes resultados partiendo de imágenes limpias, en alta resolución y con contrastes claramente definidos entre el texto y el fondo. Sin embargo, mediante la captura con cámaras debido a su naturaleza, a la forma en que se realiza la captura y el entorno en que nos encontremos se presentan una serie de dificultades que deben ser tenidas en cuenta.  
\begin{description}
\item[Baja resolución] Las imágenes obtenidas con las cámaras suelen estar en baja resolución, bien por las limitaciones del sensor, o por que la capacidad de computo del dispositivo que la contiene es limitada. Mientras que con un escáner es normal trabajas con una resolución de entre 150 a 600 dpi, el mismo texto en una captura con una cámara rodaría los 50 dpi.
\item[Iluminación no uniforme] La cámara, al contrario que el escáner no tiene control de la iluminación de la escena. En la captura mediante cámaras es normal encontrarse con iluminación no uniforme, varias fuentes de luz con temperaturas de color diferentes, sombras o reflejos que degradan la calidad de la imagen.
\item[Distorsión por perspectiva] Al capturar el texto sin estar la cámara paralela al plano en el que se encuentra el documento, se está produciendo una distorsión por perspectiva. Esto provoca que el texto presente distintos tamaños a lo largo de la imagen o que se produzca una deformación que impida el correcto reconocimiento de los caracteres. 
\item[Distorsión de la lente] Las cámara incorporadas a los teléfonos móviles suelen tener una distancia focal menor para obtener un mayor ángulo de visión. La consecuencia de esto es que la lente exagera la perspectiva de los objetos, provocando mayor distorsión en las líneas cuanto más cerca se encuentre la lente del objeto.
\item[Fondos complejos] El caso ideal para la extracción de texto es que el fondo sea totalmente uniforme y con contraste diferenciado. Una mala iluminación provocará alteraciones de tono y contraste entre texto y fondo, que dificultará la segmentación el texto. 
\item[Zoom y autoenfoque] Las cámaras actuales están equipadas con sistemas de zoom y autoenfoque. Una captura en la que existan distintos planos de profundidad o una mala iluminación provocará que el sistema de autoenfoque tenga dificultades para estabilizarse y durante ese tiempo las imágenes sean borrosas o fuera de foco.
\item[Objetos móviles] Por la propia naturaleza de los dispositivos móviles se entiende que o bien el dispositivo o el objeto a fotografiar está en movimiento (o incluso ambos). Si la velocidad de obturación de la cámara no es lo suficientemente rápida, la imagen obtenida estará movida.
\item[Ruido del sensor] Para compensar entornos con poca luz, las cámaras aumentan la sensibilidad amplificando la señal generada por las celdas del sensor. Como estos elementos tienen una emisión de señal de base mas o menos fija, al capturar una señal lumínica débil y amplificarla, estamos amplificando también una buena porción de la emisión de datos aleatoria, con lo que se mezclará una cantidad de señal aleatoria sin contenido a la señal correspondiente a la imagen. Cuanto mayor sea la amplificación, más ruido se va a generar y peor calidad de imagen vamos a obtener. 
\item[Compresión de imagen] Normalmente la imagen obtenida por el sensor se almacena comprimida mediante algoritmos con perdida de información como JPEG. La utilización de ratios altos de compresión provoca que se generen artefactos y distorsiones apreciables que restan nitidez a la imagen.
\item[Algoritmos ligeros] El objetivo final es integrar los algoritmos de análisis en los dispositivos móviles. Se deben implementar algoritmos computacionalmente eficientes ya que en la mayoría de los casos los recursos disponibles como memoria  y la capacidad de computo son limitadas.
\end{description}


\section{Impacto socio-económico}








\section{Estructura del documento}

Este documento se ha estructurado según las indicaciones de la normativa de trabajos de fin de grado de la Escuela Superior de Informática de la Universidad de Castilla-La Mancha, y contará con  los siguientes capítulos:
\begin{definitionlist}
\item[Capítulo \ref{chap:objetivos}: \nameref{chap:objetivos}] Finalidad y justificación  (con todo detalle) del presente documento.
\item[Capítulo \ref{chap:antecedentes}: \nameref{chap:antecedentes}] Explica herramientas y aspectos básicos de edición con \LaTeX.
\item[Capítulo \ref{chap:metodo}: \nameref{chap:antecedentes}] Explica herramientas y aspectos básicos de edición con \LaTeX.
\item[Capítulo \ref{chap:resultados}: \nameref{chap:antecedentes}] Explica herramientas y aspectos básicos de edición con \LaTeX.
\item[Capítulo \ref{chap:conclusiones}: \nameref{chap:antecedentes}] Explica herramientas y aspectos básicos de edición con \LaTeX.
\item[Capítulo \ref{chap:bibliografia}: \nameref{chap:antecedentes}] Explica herramientas y aspectos básicos de edición con \LaTeX.
\end{definitionlist}



% Local Variables:
%  coding: utf-8
%  mode: latex
%  mode: flyspell
%  ispell-local-dictionary: "castellano8"
% End:
