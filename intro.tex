\chapter{Introducción}

\drop{L}{a} gestión documental es una actividad que la humanidad lleva desarrollando desde hace muchos siglos. Nació prácticamente al mismo tiempo que la escritura, ante la necesidad de dar fe y registrar por escrito acuerdos comerciales o actos administrativos. A lo largo de la historia el volumen de los fondos documentales ha ido en aumento, sobre todo en el último siglo, lo que ha hecho que la tarea archivística se haya vuelto cada vez más compleja.

Según Robergé~\cite{roberge}, la gestión documental se puede definir como «\textit{el conjunto de las operaciones y de las técnicas relativas a la concepción, el desarrollo, la implantación y la evaluación de los sistemas administrativos necesarios, desde la creación de los documentos hasta su destrucción o su transferencia al archivo permanente cuyo objetivo es lograr eficiencia y economía administrativas}». Por tanto, es evidente suponer que realizando una mejora de la eficiencia en la gestión documental de una organización, vamos a obtener un aumento de la competitividad corporativa.

% La Norma ISO-15489 Internacional Standard on Records Management~\cite{ISO15489-1} aprobada en el año 2001, es el la respuesta a la normalización y uso de buenas prácticas en el proceso de organización y gestión de documentos administrativos  y  su conservación en diferentes soportes y formatos.

%\subsection{Beneficios de la gestión documental}
Entre los principales beneficios, una vez implementado en una organización un programa de gestión de documentos, se pueden encontrar \cite{zapata}:
\begin{itemize}
\item \textbf{Reducción del volumen documental:} Uno de los primeros resultados que se observan, es la disminución de la cantidad de documentos generados, sobre todo de aquellos documentos que no aportan valor informativo. 
\item \textbf{Mejoras en la eficiencia administrativa:} Una vez revisados los procesos de la organización, nos permitirá eliminar aquellos que sean redundantes, innecesarios o muy complejos, cuyo desarrollo implique la realización de tareas sin un objetivo dentro del mismo y que finalmente se derive en el aumento del tiempo de realización del proceso. 
\item \textbf{Incremento de la productividad:} %El beneficio más importante se refleja en el aumento de la productividad de los empleados.
Al producirse una reducción de la documentación, procesos como la  generación, consulta o archivo se simplifican, permitiendo que se dedique más tiempo a las labores propias del negocio.
\item \textbf{Reducción de los costos de infraestructura:} Con la estandarización se obtiene una reducción igualmente de  la infraestructura necesaria para la generación,  conservación y distribución de los documentos,  necesitándose menos equipación, espacio de almacenaje, mobiliario, etc. 
\item \textbf{Aprovechamiento de las tecnologías de la información:} La aplicación de los nuevos conceptos de gestión documental fomentan el uso de las tecnologías de la información, aplicaciones ofimáticas, automatización de tareas, etc. 
\item \textbf{Mejora de los niveles de rentabilidad de la empresa:} La reducción en costos operativos que supone la implementación de un sistema de gestión documental, permite que se pueda invertir en áreas claves del negocio.
\end{itemize}
\bigskip
En la actualidad existe un gran abanico de herramientas software con el objetivo de controlar e incrementar la eficiencia del flujo de documentos que soportan los negocios o sus actividades denominados Sistemas de Gestión Documental (SOD) o, en ingles, Document Management Systems (DMS)

\subsection{Visión por computador}

La visión por computador es un área relacionada con la  inteligencia artificial que tiene como objetivo la extracción de información del mundo físico a partir de imágenes utilizando sistemas computacionales. 

La visión artificial, en un intento de imitar o reproducir la capacidad que utiliza el ser humano para interpretar su entorno mediante el sentido de la vista, define tradicionalmente cuatro fases principales ~\cite{velez}: 

\begin{itemize}

\item \textbf{Captura:} Consiste en la adquisición de las imágenes mediante algún sensor o cámara. 

\item \textbf{Procesamiento:} Se realiza un tratamiento digital de las imágenes mediante filtros y transformaciones para eliminar o realzar partes de la imagen sobre las que se realizará el análisis posterior. 

\item \textbf{Segmentación:} En esta fase se aíslan los elementos que interesan de una escena para comprenderla. 

\item \textbf{Clasificación:} En ella se intentará distinguir los objetos segmentados realizando un análisis de ciertas características que se definen para diferenciarlos.

\end{itemize}

En la actualidad, las técnicas de visión por computador se utilizan en multitud de entornos industriales. El reconocimiento facial, seguimiento de objetos en tareas de vídeo-vigilancia, búsqueda de imágenes a través de características visuales o el mapeo de una escena para generar un modelo 3D son algunos ejemplos de exitosa implantación comercial.

La realidad aumentada se podría considerar como la aplicación de distintas técnicas de visión por computador. El principal problema que deben tratar los sistemas de realidad aumentada se considera \textbf{registro}, que consiste en calcular la posición relativa de la cámara respecto a la escena para colocar correctamente las imágenes sintéticas dentro de la  imagen real. Este registro se suele realizar empleando diferentes técnicas, como por ejemplo tracking visual~\cite{gonzalez}.

\section{Realidad Aumentada}
La realidad aumentada es una tecnología mediante la cual la percepción del mundo real se complementa con información adicional generada por ordenador en tiempo real. Esta información adicional puede ser desde etiquetas virtuales, representaciones de modelos tridimensionales, o incluso cambios de iluminación. 

%El término Realidad Aumentada fue acuñado en 1990 por Tom Caudell durante el desarrollo de un sistema de Boeing para ayudar a los trabajadores en el ensamblaje de aviones mediante la ayuda de pantallas que mostraban información del montaje. 

La realidad aumentada se puede aplicar a prácticamente todos los campos. En el sector industrial, para la reparación y mantenimiento de máquinas e instalaciones complejas, visualización de datos o simulación.  En aplicaciones médicas, mostrarían la situación de órganos no visibles durante una cirugía. También se han realizado aplicaciones para diseño de interiores (IKEA), presentaciones de productos, educación, publicidad, turismo, arte y ocio. 

%Uno de los principales problemas a resolver es denominado registro. Los objetos del mundo real y virtual deben estar correctamente alineados o la sensación de integración se verá seriamente afectada.

Las aplicaciones de Realidad Aumentada deben cumplir las siguientes características definidas por Ronald T. Azuma \cite{Azuma}:

\begin{description}
\item[Combinar el mundo real con el virtual.] El resultado final debe mostrar la información sintética sobre las imágenes percibidas del mundo real.
\item[Debe ser interactivo en tiempo real.] La integración debe ser realizada \emph{en el momento}, por lo que el cálculo necesario debe realizarse en el menor tiempo posible.
\item[La alineación de los elementos virtuales debe realizarse en 3D.] Los objetos sintéticos deben de estar correctamente alineados en el espacio tridimensional,  o la sensación de integración se verá seriamente afectada.
\end{description}
 
\subsection{Componentes de un sistema de realidad aumentada}
Un sistema de realidad aumentada depende de cuatro componentes físicos:

\begin{description}

\item[Elementos de entrada y sensores de orientación] Proporciona al sistema la información visual, la entrada del usuario y, potencialmente, ayudar a la orientación.

\item[Fuente de datos] Proporciona información aplicable al medio ambiente para que el sistema aumente la visión del usuario.

\item[Periféricos de retroalimentación de los usuarios] Principalmente en la forma de producción visual, pero también pueden incluir audio y otras interfaces de usuario.

\item[Unidad de procesamiento] Combina los datos de los sensores de entrada para determinar la orientación y aumentar la visión del usuario con información de la fuente de datos. Envía el resultado a los periféricos de retroalimentación del usuario.
\end{description}

\subsubsection{Elementos de entrada y sensores de orientación}
Un sistema de realidad aumentada debe ser capaz de conocer su orientación y posición dentro del entorno circundante con el fin de ofrecer una experiencia ``aumentada'' satisfactoria. La combinación de la orientación y la posición de un observador que se conoce como \emph{pose}. La estimación de la \emph{pose} se definirá con más detalle en la sección 4. La orientación y posición se pueden determinar mediante técnicas de visión por computador, pero hay métodos alternativos para estimar la \emph{pose}  utilizando  sensores diseñados específicamente para medir alguno o todos los parámetros de los seis grados de libertad que componen la \emph{pose}.

\paragraph{Localización}
La manera más popular para determinar la posición es mediante la utilización del Sistema de Posicionamiento Global (GPS), desarrollado por el ejército de Estados Unidos y totalmente operativo desde 1994. 
Un receptor GPS es lo suficientemente preciso para determinar la ubicación en las aplicaciones de realidad aumentada, además, debido a su bajo coste, los sensores de GPS se llevan incluido en los teléfonos móviles desde 2005.

\paragraph{Orientación}
En el espacio 3D, la orientación de un cuerpo rígido puede ser determinado de forma única por tres ángulos o grados de libertad (3DOF): \emph{pitch, roll y yaw}. La figura \ref{fig:ángulos}  se puede observar una representación de estos tres ángulos como rotaciones alrededor de sus ejes correspondientes. 

Para facilitar la tarea de orientación los dispositivos móviles, sobre todo smartphones y tablets, están equipados una serie de sensores como:
\begin{itemize}
\item Sensores magneto-resistivos, o magnetómetros, que son utilizados como una ``brújula electrónica'' y se utilizan para determinar el \emph{yaw} y la dirección. Las nuevas generaciones de brújulas electrónicas utilizan tres ejes para asegurar una lectura correcta sin importar la inclinación del sensor.
 
\item Acelerómetros para medir la aceleración aceleración experimentada por un objeto con respecto a un marco de referencia. Los acelerómetros no pueden determinar el \emph{yaw}, pero pueden ser utilizados para determinar el \emph{pitch} y el \emph{roll}. 
\item Giroscopios electrónicos que son utilizados para determinar los tres grados de libertad de orientación a la vez.
\end{itemize}

Un problema común en los sistemas de seguimiento visual es que los movimientos bruscos de la cámara pueden ocasionar que la imagen salga borrosa y verse afectado el seguimiento visual. Un dispositivo híbrido equipado con un giroscopio y un acelerómetro será capaz de determinar rápidamente y con precisión tres de los seis grados de libertad de la \emph{pose} y ayudar al sistema de seguimiento visual a recuperarse cuando se experimentan imágenes borrosas. 

\paragraph{Fuentes de Vídeo}
La realidad aumentada superpone información sobre lo que el usuario está viendo. Para los sistemas dinámicos en los que la información virtual debe ser mezclada con la imagen real, se requiere un dispositivo de captura, como una cámara digital. Para los propósitos de las aplicaciones de realidad aumentada es preferible una cámara de vídeo. La mayoría de las aplicaciones utilizan una única cámara y se conocen como sistemas monoculares. También existen sistemas estereoscópicos, que  utilizan dos cámaras para determinar la profundidad.

El desarrollo tecnológico en el ámbito de la captura de imagen digital que han mantenido los principales fabricantes en los últimos 10 años, conocido como la batalla del mega-píxel, ha supuesto un gran avance para disponer de sensores con una gran calidad de imagen y un coste reducido. Actualmente es normal encontrar cámaras de alta resolución en prácticamente todos los dispositivos móviles, permitiendo grabar vídeo en una resolución de 1920x1080 (FullHD) y obtener imágenes fijas de hasta 7136x5360px.

\paragraph{Entrada táctil}
En sistemas de realidad aumentada portátiles tales como teléfonos inteligentes o tablets, la pantalla funciona al mismo tiempo como una entrada para manipular objetos directamente en la pantalla y un dispositivo de salida.

\subsubsection{Fuente de datos}
Un sistema de realidad aumentada requiere una fuente de datos e información con la que aumentar la realidad de un usuario. La fuente de datos puede ser desde una base de datos directamente disponible en el sistema, o puede ser información agregada a partir de fuentes disponibles a través de una conexión de red. Actualmente existe un gran desarrollo para el uso de Internet como fuente de datos para aplicaciones de realidad aumentada. Aplicaciones como Wikitude y Layar son ejemplos de las sistemas que utilizan los datos disponibles en el Internet como fuente de datos.

\subsubsection{Periféricos de retroalimentación de los usuarios}
\paragraph{Dispositivos  de Visualización}
Existen  muchos métodos y dispositivos para mostrar la información o imágenes generadas para su combinación y alineación con la realidad: 
\begin{description}
\item[Monitores y pantallas tradicionales]
  Sobre todo impulsado por la popularidad de frameworks y bibliotecas como ARToolkit \cite{Kato}, existen muchas implementaciones de realidad aumentada para utilizar con un ordenador de sobremesa o portátil. No requieren de tecnología especial, únicamente una webcam que el usuario puede controlar. La aplicación se suele presentar en una pagina web y permite a los usuarios superponer información o imágenes sobre lo capturado por la webcam. Para la tarea de tracking y registro se utilizan marcadores que el usuario previamente ha impreso.
  
\item[HMDs (Head Mount Display)] Los HDM o Head Mount Display son dispositivos que permiten al usuario ver el el mundo real con objetos virtuales superpuesto mediante técnicas ópticas y de vídeo. Los HDM se pueden clasificar en dos categorías:
  \begin{description}
  \item[Ópticos (OST - Optical See-Throug)] Permiten al usuario ver el mundo real a través de sus ojos y la imagen virtual se muestra mediante elementos ópticos holográficos, espejos semi-plateados o alguna otra técnica similar. Un ejemplo de este tipo de dispositivo son las Google Glass.
    
  \item[Vídeo (VST - Vídeo See-Throug)] En los dispositivos de esta categoría, el usuario percibe tanto el mundo real como la imagen virtual mediante una o varias pantallas. Como ventajas de los VST esta una mayor consistencia entre el mundo real y el virtual.
  \end{description}
  
\item[Visualización basada en proyección]
  Un sistema de visualización basado en proyección es una buena opción para sistemas fijos, además proporciona una intrusión mínima y la capacidad de interacción con varios usuarios.
  Se han propuesto una gran variedad de técnicas de visualización mediante proyectores sobre objetos y otras superficies (no necesariamente pantallas). Ehnes \cite{Ehnes} amplió el trabajo anterior de Pinhanez \cite{Pinhanez} sobre la proyección de vídeo para mostrar imágenes virtuales sobre objetos reales directamente utilizando técnicas de tracking de vídeo para seguir el movimiento y manteniendo la imagen sobre objeto mientras este se mueve.
  
\item[Dispositivos portátiles (smartphones, tablets)]
  Hoy en día el dispositivo portátil por excelencia es el smartphone. Es mínimamente invasivo, muy portable y ampliamente extendido. Según el informe de comScore \cite{comScore} el 66\% de los españoles tienen un teléfono inteligente, por lo que son una gran alternativa para la visualización de aplicaciones de realidad aumentada.
\end{description}

\paragraph{Audio}
El audio se utiliza principalmente de dos formas diferentes en los sistemas de realidad aumentada: como una modalidad de la interfaz de usuario multimodal o para ofrecer una experiencia aural aumentada. 

Por ejemplo, el usuario pueda ser capaz de dar comandos de voz y el sistema devuelva retroalimentación con señales de audio (pitidos). Este tipo de audio no direccional es trivial desde el punto de vista tecnológico aportando una nueva dimensión a las aplicaciones móviles. El audio aumentado puede servir a personas con discapacidad visual para ayudar a comprender el entorno. Mediante unos altavoces o un par de auriculares, es bastante simple proporcionar al usuario la información en forma de audio. La síntesis de voz también se puede utilizar para leer información, por ejemplo, leer palabras en voz alta desde un sistema de reconocimiento de texto.

\subsubsection{Unidad de procesamiento}
\paragraph{Dispositivos móviles: Smartphones,  tablets y SBCs (Single Board Computer)}

El teléfono móvil ha evolucionado hacia una plataforma informática móvil (smartphone), en la que desarrollar las actividades habituales de un ordenador personal. El principal inconveniente de estos era la pantalla reducida que disponen (hasta 5”) lo que ha derivado en el desarrollo de un nuevo dispositivo denominado tablet. 

Las tablets disponen de prácticamente las mismas características que los smartphones (incluso el mismo sistema operativo) pero las pantallas son mayores (hasta 10”) sin necesidad de teclado físico ni ratón, con las que se interactúa principalmente mediante la pantalla táctil con los dedos o un stylus (Samsung Galaxy Note).

Otra tendencia tecnológica es el desarrollo de SBCs, (Single Board Computer) que básicamente es un ordenador completo montado sobre un circuito. Su diseño se basa en un microprocesador de bajo rendimiento (normalmente de arquitectura ARM), RAM y diversos dispositivos de E/S como puertos USB, lectores de tarjetas, conectores Ethernet, y salida de audio y vídeo con un precio final muy reducido (alrededor de 40 \euro). Este tipo de dispositivos se crearon para utilizarlos con fines educativos, pero entre la comunidad de usuarios, debido a su gran versatilidad se utilizan para múltiples aplicaciones tales como centros multimedia o distintos tipos de servidores dado su bajo consumo.  

\section{Reconocimiento y análisis de documentos}

\subsection{Texto y documentos}
En general, podemos considerar que cualquier escena o imagen que tenga un contenido textual como si fuera un documento. Esto incluiría tanto un libro, la matricula de un vehículo o un cartel en una pared. La mayoría de trabajos mediante cámaras están basados en la extracción de texto en imágenes fijas o secuencias de vídeo en las que los autores denominan imágenes naturales, en lugar de imágenes donde el texto está estructurado como los documentos. Ambos enfoques tienen sus desafíos y distintos modos de acometerlos, pero el objetivo final de todos es la de dotar a las cámaras la capacidad de lectura.

En el caso de documentos estructurados, que es el del dominio de este trabajo, las imágenes suelen ser documentos impresos como artículos, cartas, formularios o páginas de libros, donde gran parte de la imagen se asume que es texto, pero también puede contener figuras, diagramas e incluso algunos autores han tratado con anotaciones escritas a mano \cite{Chen}

\subsection{Identificación y recuperación de documentos}
Aunque hoy en día la mayor parte de la producción de información en forma de documentos se realiza por medio de herramientas informáticas (procesadores de texto, correo electrónico, etc.), puede ocurrir, y de hecho será un caso muy habitual, que la información no se restrinja a documentos actuales, ya automatizados, sino que se encuentre impresa. Incluso puede ocurrir que nos interese disponer sólo de la información antigua (archivos y manuscritos).

En estos casos para conseguir una gestión eficaz y ágil, es necesario digitalizar previamente estos documentos para incorporarlos al sistema que tenga implementado la organización.

Las primeras aplicaciones se basaban en el paradigma de reconocimiento de caracteres (OCR), donde se utilizaban estas técnicas para realizar un análisis del contenido informativo de los documentos y utilizarlo para su clasificación y almacenamiento. 

La recuperación de objetos (también nombrada por otros autores reconocimiento o identificación) se incorpora recientemente en la detección de tal manera que un objeto es capturado en una imagen, recuperado de una base de datos y su pose inicial se calcula simultáneamente \cite{Pilet}

El desarrollo de la investigación realizada en este ámbito se inició con los métodos que utilizaban marcas especiales en el documento, como códigos de barras \cite{Graham} o glifos para vincular contenido electrónico con las imágenes capturadas \cite{Hecht}. Los inconvenientes de estos enfoques es que es necesario modificar el formato y la apariencia del documento para introducir las marcas, que en algunos casos pueden distraer al usuario del contenido del documento. Por otro lado, un documento válido para el sistema al que no se le hayan incluido previamente estas marcas, no será detectado y vinculado con la información a recuperar.

La utilización del teléfono móvil y otros dispositivos portátiles para la identificación de documentos ha  que publiquen diversos artículos con algoritmos y métodos que parten de las propias limitaciones de estos dispositivos como es la baja capacidad de computo, la calidad de las imágenes capturadas, en muchos casos borrosas, y la captura parcial del documento.

\section{Limitaciones de la realidad aumentada en la identificación de documentos}
\subsection{Adquisición de imágenes mediante cámaras. Ventajas e inconvenientes}
El análisis de documentos mediante cámaras tiene una serie de ventajas sobre aquellos que están basados en la adquisición mediante escáner. Las cámaras son pequeñas y fáciles de transportar. También se pueden utilizar en cualquier entorno y sobre documentos que por su formato sean difíciles de manipular en un escáner como periódicos, libros, o manuscritos antiguos. Incluso para capturar texto que no se encuentra en papel, como carteles en fachadas, o texto que se encuentre el objetos que se muevan por la escena.

En la mayoría de casos, los escáneres obtienen mejor calidad en la captura de calidad que las realizadas mediante cámaras, pero los sistemas basados en cámaras son mucho más flexibles y portables. 

\subsection{Problemática asociada}
Casi todos los algoritmos de reconocimiento de documentos obtienen grandes resultados partiendo de imágenes limpias, en alta resolución y con contrastes claramente definidos entre el texto y el fondo. Sin embargo, mediante la captura con cámaras debido a su naturaleza, a la forma en que se realiza la captura y el entorno en que nos encontremos se presentan una serie de dificultades que deben ser tenidas en cuenta.  
\begin{description}
\item[Baja resolución] Las imágenes obtenidas con las cámaras suelen estar en baja resolución, bien por las limitaciones del sensor, o por que la capacidad de computo del dispositivo que la contiene es limitada. Mientras que con un escáner es normal trabajas con una resolución de entre 150 a 600 dpi, el mismo texto en una captura con una cámara rodaría los 50 dpi.
\item[Iluminación no uniforme] La cámara, al contrario que el escáner no tiene control de la iluminación de la escena. En la captura mediante cámaras es normal encontrarse con iluminación no uniforme, varias fuentes de luz con temperaturas de color diferentes, sombras o reflejos que degradan la calidad de la imagen.
\item[Distorsión por perspectiva] Al capturar el texto sin estar la cámara paralela al plano en el que se encuentra el documento, se está produciendo una distorsión por perspectiva. Esto provoca que el texto presente distintos tamaños a lo largo de la imagen o que se produzca una deformación que impida el correcto reconocimiento de los caracteres. 
\item[Distorsión de la lente] Las cámara incorporadas a los teléfonos móviles suelen tener una distancia focal menor para obtener un mayor ángulo de visión. La consecuencia de esto es que la lente exagera la perspectiva de los objetos, provocando mayor distorsión en las líneas cuanto más cerca se encuentre la lente del objeto.
\item[Fondos complejos] El caso ideal para la extracción de texto es que el fondo sea totalmente uniforme y con contraste diferenciado. Una mala iluminación provocará alteraciones de tono y contraste entre texto y fondo, que dificultará la segmentación el texto. 
\item[Zoom y autoenfoque] Las cámaras actuales están equipadas con sistemas de zoom y autoenfoque. Una captura en la que existan distintos planos de profundidad o una mala iluminación provocará que el sistema de autoenfoque tenga dificultades para estabilizarse y durante ese tiempo las imágenes sean borrosas o fuera de foco.
\item[Objetos móviles] Por la propia naturaleza de los dispositivos móviles se entiende que o bien el dispositivo o el objeto a fotografiar está en movimiento (o incluso ambos). Si la velocidad de obturación de la cámara no es lo suficientemente rápida, la imagen obtenida estará movida.
\item[Ruido del sensor] Para compensar entornos con poca luz, las cámaras aumentan la sensibilidad amplificando la señal generada por las celdas del sensor. Como estos elementos tienen una emisión de señal de base mas o menos fija, al capturar una señal lumínica débil y amplificarla, estamos amplificando también una buena porción de la emisión de datos aleatoria, con lo que se mezclará una cantidad de señal aleatoria sin contenido a la señal correspondiente a la imagen. Cuanto mayor sea la amplificación, más ruido se va a generar y peor calidad de imagen vamos a obtener. 
\item[Compresión de imagen] Normalmente la imagen obtenida por el sensor se almacena comprimida mediante algoritmos con perdida de información como JPEG. La utilización de ratios altos de compresión provoca que se generen artefactos y distorsiones apreciables que restan nitidez a la imagen.
\item[Algoritmos ligeros] El objetivo final es integrar los algoritmos de análisis en los dispositivos móviles. Se deben implementar algoritmos computacionalmente eficientes ya que en la mayoría de los casos los recursos disponibles como memoria  y la capacidad de computo son limitadas.
\end{description}


\section{Impacto socio-económico}

El presente proyecto se enmarca dentro de la Cátedra Indra-UCLM, en el proyecto “ARgos: Sistema de Ayuda a la Gestión Documental basado en Visión por Computador y Realidad Aumentada” que tiene como objetivo la construcción de un sistema de ayuda a la gestión de documentos, basado principalmente en visión por computador y síntesis visual y auditiva en el espacio físico, empleando técnicas de realidad aumentada.  

El proyecto Argos está pensado para facilitar la integración laboral de cualquier persona con discapacidad que tenga que gestionar documentación impresa. 

Todos los países de la Unión Europea aceptan las orientaciones generales de la Organización Mundial de la Salud así como las directrices y programas de las Naciones Unidas relativas a las personas con discapacidades. En especial, las políticas nacionales de los años ochenta y noventa tomaron como principal referencia el Programa de Acción Mundial para los Impedidos, aprobado por la Asamblea General de las Naciones Unidas en 1982 y que proponía expresamente «la participación plena de los individuos con discapacidad en la vida social, con oportunidades iguales a las de toda la población». (10) Sin embargo, las estadísticas de los diversos países presentan conceptos y metodologías diferentes que hacen muy difícil la comparación internacional y la planificación de políticas generales comunes. La propia Comisión Europea ha planteado la necesidad de una información estadística y demográfica, elaborada con criterios homogéneos, que permita conocer la prevalencia de personas con discapacidad en los países de la Unión y, en particular, su grado de inserción en el mercado laboral.

La inserción laboral de las personas con discapacidad presente en España el perfil más negativo entre todos los países europeos, seguida de cerca por Irlanda e Italia (gráfico 1.2). De las personas con discapacidad severa, sólo tiene trabajo remunerado el 13,1\% en España, el 13,7\% en Irlanda y el 15,3\% en Italia; entre quienes tienen discapacidad moderada, la peor posición corresponde a Irlanda (27,1\%), seguida por España (28,7\%) e Italia (29,6\%). Se trata de tres países de la Unión con bajas tasas de actividad en la población general, lo que explica en parte que sean también los que menos fomentan el acceso al empleo de las personas con discapacidades.

La media comunitaria de personas con discapacidad severa que tienen empleo es del 24,3\%. Los países con tasas más altas son Francia y Portugal, precisamente los más próximos al territorio español. España, además, es junto con Italia el país europeo que presenta una mayor discriminación de género en el acceso al empleo: las mujeres con discapacidad severa no
sólo son las que registran una menor proporción de ocupadas (9\%), sino las que se encuentran a mayor distancia en términos relativos de la tasa de hombres ocupados del propio país (16,6\%). En el conjunto de la Unión tienen
empleo el 27,9\% de los hombres con discapacidad severa y el 20,5\% de las mujeres. Sólo en tres países, Dinamarca, Reino Unido y Grecia, las mujeres están más ocupadas que los hombres (tabla 1.5 del Anexo estadístico).

Grado de severidad y pronóstico evolutivo de las discapacidades. La mayoría de las personas afectadas en edad laboral puede
trabajar
Podemos partir de las personas en edad laboral que presentan discapacidades para las actividades de la vida diaria, para analizar su grado de severidad:

La discapacidad es total en el 28,7\% de los casos, de manera que están impedidas absolutamente para realizar las acciones correspondientes.

Esto no quiere decir que tales personas no puedan trabajar, ya que disponen casi siempre de otras facultades que les habilitan para desarrollar una amplia gama de empleos. Por ejemplo, si no pueden manejar con soltura manos y dedos, pueden escoger un oficio en el que no se exija esa habilidad.

La severidad es grave en un tercio de los casos, pudiendo entonces desarrollar la actividad correspondiente, aunque sea con dificultad o con menor eficiencia.

Por último, cuando la severidad es moderada, lo que ocurre en el 37,4\% de los casos, las personas con discapacidad pueden realizar las acciones correspondientes sin mayores dificultades, normalmente gracias a la ayuda de aparatos o de procesos específicos de rehabilitación.

Dos de cada tres discapacidades no impiden realizar la actividad correspondiente, siempre que se disponga de las ayudas técnicas y los medios de rehabilitación adecuados.
 

La estructura de los mercados de trabajo y su evolución actual es otro de los contextos decisivos para explicar las trayectorias de inserción social y laboral de las personas con discapacidades en España. Las tasas de actividad, ocupación y paro, la diversa calidad de los empleos existentes, las exigencias de cualificación y eficiencia en el trabajo, etc., representan otras tantas condiciones estructurales que delimitan tanto las posibilidades como las barreras para la inserción laboral del colectivo estudiado. 

La evolución del mercado de trabajo español en las últimas décadas se caracteriza en general por un incremento de la competitividad y la segmentación entre los trabajadores, que, si bien ha permitido aumentar significativamente la productividad, ello ha sido a costa de una notable fragmentación del mercado laboral en el que coexisten dos tendencias diferentes en relación a la mano de obra:

Políticas de gestión «positivas» para los empleos más cualificados o de los que depende la estabilidad de las empresas o de los servicios de la administración pública (el llamado sector primario del mercado labo-



\section{Estructura del documento}

Este documento se ha estructurado según las indicaciones de la normativa de trabajos de fin de grado de la Escuela Superior de Informática de la Universidad de Castilla-La Mancha, y contará con  los siguientes capítulos:
\begin{definitionlist}
\item[Capítulo \ref{chap:objetivos}: \nameref{chap:objetivos}] Finalidad y justificación  (con todo detalle) del presente documento.
\item[Capítulo \ref{chap:antecedentes}: \nameref{chap:antecedentes}] Explica herramientas y aspectos básicos de edición con \LaTeX.
\item[Capítulo \ref{chap:metodo}: \nameref{chap:metodo}] Explica herramientas y aspectos básicos de edición con \LaTeX.
\item[Capítulo \ref{chap:resultados}: \nameref{chap:resultados] Explica herramientas y aspectos básicos de edición con \LaTeX.
\item[Capítulo \ref{chap:conclusiones}: \nameref{chap:conclusiones}] Explica herramientas y aspectos básicos de edición con \LaTeX.
\item[Capítulo \ref{chap:bibliografia}: \nameref{chap:bibliografia}] Explica herramientas y aspectos básicos de edición con \LaTeX.
\end{definitionlist}



% Local Variables:
%  coding: utf-8
%  mode: latex
%  mode: flyspell
%  ispell-local-dictionary: "castellano8"
% End:
