\chapter{Manual de calibración}

El proceso de calibrado del sistema se ha creado como una utilidad independiente del sistema principal de ARgos. Una vez calibrado el sistema, la aplicación proporciona los ficheros XML, con los parámetros intrínsecos y extrínsecos, que necesita ARgos para el calculo del registro.

El proceso de calibrado de la cámara esta basado esencialmente por el enfoque de Zhang. Se utiliza un patrón tipo tablero de ajedrez, en la que se alternan cuadrados blancos y negros, de dimensiones conocidas. El patrón se imprime y se pega sobre una superficie plana rígida. Se debe dejar una zona despejada a continuación del tablero de ajedrez, ya que en esta zona se proyectará el patrón que utiliza el proyector para su calibrado.  

A continuación, colocamos la superficie bajo la cámara, en distintas orientaciones y distancias, para obtener una serie de imágenes en los que se encuentre visible el patrón. 

Cuando el proceso disponga de al menos 20 imágenes válidas, realizará los cálculos para obtener los parámetros intrínsecos de cámara y los grabará en el fichero \texttt{calibrationCamera.xml}

El siguiente paso consiste en calibrar el proyector. Se proyecta un patrón de círculos asimétrico, en una posición fija. Colocar la superficie de tal forma que el patrón proyectado quede junto al del tablero de ajedrez. Volver a mover el patrón por espacio de proyección hasta que el sistema adquiera 5 imágenes. En ese momento, la calibración pasa al modo dinámico, y los puntos proyectados se generan en función de la posición que tenga tablero de ajedrez. 

Tras capturar 15 imágenes con los puntos dinámicos, se realizan los cálculos de los parámetros intrínsecos del proyector, y finalmente, de los parámetros extrínsecos (matrices de rotación y translación) entre la cámara y el proyector. 

Los parámetros de calibrado se almacenan en los ficheros \texttt{calibrationProjector.xml} y \texttt{cameraProjectorExtrinsics.xml}.

Mientras que la cámara y el proyector mantengan su posición y rotación entre ellos, no es necesario realizar una nueva calibración y es posible mover todo el sistema.

%\blindtext
