\chapter{Resultados}
\label{chap:resultados}




\section{Arquitectura}
\subsection{Descripción general}
\subsection{Módulo de captura}
El módulo de captura de vídeo se encarga de crear y proporcionar fuentes de vídeo de diversa naturaleza para dar soporte al submódulo de registro (ver Sección 5.5), y al submódulo de representación 2D (ver Sección 5.3.4).
Es capaz de dar soporte multicámara. Se pueden crear tantas fuentes de vídeo como se disponga en el sistema, y el submódulo de vídeo las clasificará como: 
\begin{itemize}
\item RaspiCam: es necesaria esta fuente de vídeo para ejecutar una aplicación Minerva. Realiza el registro de la realidad y se dibuja de fondo.
\item Cámara USB: no es necesario que exista ninguna. Sirven de soporte para realizar registro sobre ellas, pero a priori no se visualiza al usuario.
\end{itemize}

El submódulo de vídeo se implementa en dos componentes: el VideoFactory, que proporciona la interfaz de creación y gestión de fuentes de vídeo, y el VideoSource, que implementa una única fuente de vídeo basáda en la biblioteca de visión artificial OpenCV.

OpenCV utiliza la clase cv::VideoCapture como abstracción de las de fuentes de vídeo a las que da soporte.
Para la realización de este proyecto se ha compilado una versión de OpenCV con soporte de archivos de vídeo basado en FFMPEG y con soporte de dispositivos de vídeo basado en Video4Linux 2. Así, la idea inicial de utilizar OpenCV para encapsular el acceso a cualquier tipo de fuente de vídeo 
\subsubsection{Biblioteca RaspiCam}



\subsection{Módulo de calibración}
Como ya se explicó en el capitulo \ref{chap:antecedentes}, los objetivos de realizar una calibración son dos: obtener los
parametros intrinsecos y extrinsecos de la cámara y el proyector y calcular los coeficientes de distorsión para ajustarse al modelo de camara pinhole ideal.

El módulo implementado está basado en una extensión que han realizado Alvaro Cassinelli y Niklas Bergström a partir de un complemento de calibración desarrollado por Kyle McDonald que es capaz de calibrar cámaras y proyectores, consigiendo parametros intrínsecos de ambos, además de los extrinsecos en cuestión de varios minutos 

Se ha creado como una utilidad a parte del proceso principal. Ya que una vez calibrado el sistema, se proporcionan unos ficheros XML con los parametros intrinsecos y extrinsecos que se cargan en el proyecto. Mientras que la cámara y el proyector mantengan su posición y rotacion entre ellos, no es necesario realizar una nueva calibración y es posible mover todo el sistema.

Aunque la cámara y el proyector podrían ser calibrados de forma simultánea, es mejor comenzar primero por calibrar la cámara. 
 
El proceso consiste en obtener una serie de 





A continuación, la cámara se utiliza para calcular la posición 3D de los círculos proyectados (por retroproyección). Entonces podemos empezar a calibrar el proyector. Una vez que tengamos una buena error reproyección suficiente (para el proyector), podemos empezar a mover los puntos proyectados en torno a fin de explorar mejor el espacio (y obtener una mejor y más precisa calibración). En esta fase, podemos ejecutar la calibración estéreo OpenCV para obtener los parametros extrinsecos cámara-proyector. Después de unos pocos ciclos (y limpieza de malas "tablas"), el proceso converge, los datos se guardan y se puede hacer AR (*) 

Básicamente, el procedimiento se divide en tres pasos: 

\subsubsection{Calibración de la cámara}

1) se inicia por primera calibración de la cámara, y el cálculo de sus parametros intrínsecos (el programa es muy similar a la de Kyle McDonald ofxCV: https://github.com/kylemcdonald/ofxCv) 

\subsubsection{Calibración del proyector}

 Notes: In case of camera/projector calibration, we need to proceed in TWO PHASES to give time to the projected image to refresh before trying to detect it (in case of "dynamic projected pattern"). Otherwise we may be detecting the OLD projected pattern, but using the newer image points (which completely breaks the calibration of course)
                
PHASE 1 goal is just to set the projector image points to be projected, so the projector  will project something to be detected in the draw function. The first time, this is using the recorded pattern, but later (as the projector gets calibrated) we can use some arbitrary points "closer" to the printed pattern.

In this later case (dynamic pattern), if the printed pattern is not visible, then we won't go to phase 2. 
                

En caso de calibración de la cámara / proyector, tenemos que proceder en dos fases para dar tiempo a la imagen proyectada para refrescar antes de tratar de detectarlo (en caso de "patrón proyectado dinámica"). De lo contrario podemos estar detectando el viejo patrón proyectado, pero el uso de los puntos de imagen más recientes (que rompe por completo la calibración por supuesto) 

FASE 1 gol es sólo para establecer los puntos de la imagen del proyector para ser proyectados, por lo que el proyector proyectará algo para ser detectado en la función draw. La primera vez, esto es usar el patrón grabado, pero más tarde (ya que el proyector se calibra) podemos utilizar algunos puntos arbitrarios "más cerca" del patrón impreso.



El proceso de calibrado del proyector se divide en dos fases










el proyector proyecta una rejilla de círculos (primero en una posición fija, a continuación, como la calibración logra cierta exactitud, la parrilla de salida siguiendo el patrón impreso). Este paso es el siguiente: 

(a) la cámara se utiliza para calcular la posición 3D de los círculos proyectados (por retroproyección) en la cámara de sistema de coordenadas, y luego en el "tablero" de coordenadas del sistema de coordenadas (o "mundo"). Esto se hace en el método de "backProject" (método de la calibración clase, pero llama sólo cuando el objeto es de tipo "cámara"). 

(b) Esto significa que podemos comenzar a computar los instrinsics del proyector porque tiene puntos 3d (los círculos proyectados) en coordenadas del mundo, y sus respectivos "proyección" "imagen" en el proyector plano. 

(c) Finalmente, podemos empezar a calcular los extrinsics de la cámara-proyector (porque básicamente tenemos puntos 3d en "coordenadas mundo" (la mesa), que son los círculos proyectados, y también su proyección (puntos 2d) en la cámara imagen y proyector "imagen"). Esto significa que puede utilizar la rutina de calibración estéreo estándar en OpenCV (esto se hace en la función. La razón por la cual se calcula en primer lugar el instrinsics del proyector se debe a que el método stereoCalibrationCameraProjector llamará a la función de calibración OpenCV estéreo utilizando "intrínsecos fijos" para asegurar mejor convergencia del algoritmo (presumiblemente mejor, porque si la cámara está muy bien calibrado en primer lugar, a continuación, los instrinsics del proyector debe ser bueno - probablemente mejor que si recalculado desde todos los puntos 3D y puntos de imagen en la cámara y el proyector). 

3) Una vez que tengamos una buena error reproyección suficiente (para el proyector), podemos empezar a mover los puntos proyectados en torno a fin de explorar mejor el espacio (y obtener una mejor y más precisa calibración). En esta fase, podemos ejecutar la calibración estéreo OpenCV para obtener los extrinsics cámara-proyector y otra vez, no necesitamos para volver a calcular los instrinics del proyector. Después de unos pocos ciclos (y limpieza de malas "tablas"), el proceso converge, los datos se guardan y se puede hacer AR. 

\subsubsection{Cálculo de las matrices de transformación camara-proyector}
\subsection{Módulo de tracking y registro}
\subsection{Módulo de detección de documentos}
\subsection{Módulo de interaccion natural}
\subsection{Módulo de soporte y utilidades}
\subsubsection{Gestor de configuración}
Existen muchos parametros configurables en GrayAR. En las primeras versiones del proyecto estos parametros estaban implementados en variables dentro del programa y cada vez que era necesario modificar estos parametros, por ejemplo para la realizacion de pruebas, era necesario recompilar el código. Para los archivos que se encuentran en la raspberry, el tiempo de compilación al modificar el valor de uno de esto parametros podia incluso a tardar varios minutos. Otro problema que surgió al crecer el proyecto esra que estos parametros estaban repartidos entre varios ficheros, lo que suponia tener que recordar donde estaba localizado cada parametro y el consecuente riesgo de dejarse alguno sin actualizar que invalidaria las pruebas realizadas.

Lo primero que se realizo fue sacar todos los parametros configurables a un fichero XML que es leido al inicio del programa permitiendo que todos los parametros sean accesibles por cualquier módulo dentro del programa.

Las ventajas son apreciables a instante, el proceso de pruebas es mucho más rápido, ya que no es necesario volver a compilar cada vez que se realiza un cambio en la configuración. Todos los parametros se encuentran en un único fichero, que al ser XML, tiene una estructura clara y legible para las personas, y que permite una edicion y modificacion sencilla. Tambien permite tener varios ficheros con distintas configuraciones y que se cargue en el programa uno u otro en funcion de las necesidades del momento.
 
La necesidad es la de tener una clase centralizada y accesible desde cualquier parte del programa que sea capaz de leer un fichero de configuración en XML y almacene aquellas variables parametrizables de

Para implementar el gestor de configuración se ha creado la clase ConfigManager. Esta tiene la funcion loadConfiguration a la que se le indica el fichero xml con la configuración a cargar 

Se ha optado por seguir el principio de diseño KISS que establece que la mayoría de sistemas funcionan mejor si se mantienen simplesen lugar de hacerlos complejos; por ello, la simplicidad debe ser mantenida como un objetivo clave del diseño, y cualquier complejidad innecesaria debe ser evitada. Las ventajas genéricas que ofrece un patron Singleton frente a la implementación de un clase estática como puede ser que las funciones se puedan sobreescribir en las clases herederas, que la inicialización asincrona mientras que una clase estática generalmente se inicializa cuando se carga por primera vez o que el singleton pueda ser manejado polimorficamente sin forzar a los usuarios a asumir que solo existe una instancia, no suponen un mejora para la funcionalidad que necesitamos implementar. 

\subsubsection{Funciones de representación para depuración}
Aunque la creación de funciones para la representación gráfica están fuera del alcance de este proyecto, a la hora de probar y depurar los distintos algoritmos implementados resulta muy practico disponer de algunas funciones de dibujado ya implementadas para corregir errores y tener una vision preeliminar de como puede quedar finalmente.

A partir de las primitivas de dibujado que ofrece OpenCV se han implementado una serie de funciones para facilitar la representación de contornos, esquinas, y una vez obtenidos los parametros extrinsecos de las camara y el proyector, dibujado de ejes, cubos o cajas 3D que permiten validar si los cálculos se han realizado correctamente.


