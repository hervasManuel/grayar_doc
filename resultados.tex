\chapter{Resultados}
\label{chap:resultados}


En este capítulo se mostrarán los resultados obtenidos al aplicar la metodología descrita en la sección \ref{chap:metodo} a través de un enfoque top-down, comenzando con una descripción general, continuando con las decisiones
de diseño y terminando con los detalles de implementación. 

\section{Descripción general}
\section{Módulo de captura}
El módulo de captura de vídeo se encarga de crear y proporcionar fuentes de vídeo de diversa naturaleza para dar soporte al submódulo de registro (ver Sección 5.5), y al submódulo de representación 2D (ver Sección 5.3.4).
Es capaz de dar soporte multicámara. Se pueden crear tantas fuentes de vídeo como se disponga en el sistema, que será de alguno de los siguientes tipos: 
\begin{itemize}
\item RaspiCam: Para un redimiento óptimo en la raspberry pi, al menos la cámara principal debería ser de este tipo.
\item Cámara USB: En caso de no disponer de una raspicam o si se quieren incluir cámara adicionales para obtener capturas desde otra posición (videoconferencia,...).
\end{itemize}

El submódulo de vídeo se implementa en dos componentes: el VideoFactory, que proporciona la interfaz de creación y gestión de fuentes de vídeo, y el VideoSource, que implementa una única fuente de vídeo basáda en la biblioteca de visión artificial OpenCV.

OpenCV utiliza la clase cv::VideoCapture como abstracción de las de fuentes de vídeo a las que da soporte.
Para la realización de este proyecto se ha compilado una versión de OpenCV con soporte de archivos de vídeo basado en FFMPEG y con soporte de dispositivos de vídeo basado en Video4Linux 2. Así, la idea inicial de utilizar OpenCV para encapsular el acceso a cualquier tipo de fuente de vídeo 

\subsection{Biblioteca RaspiCam}
La raspicam no es compatible con \emph{video4linux} y se debe utilizar la API MMAL (Multi-Media Abstraction Layer) sobre OpenMAX, para acceder a los datos de la cámara y transferirlos a la pantalla o codificarlo como imágenes o vídeos.
El grupo de investigación "Aplicaciones de la Visión Artificial" de la Universidad de Córdoba ha liberado una biblioteca en C++, que haciendo uso de la API anterior, porporciona una interfaz sencilla, sin necesidad de dependencias y compatible con los objetos de OpenCV.

El rendimiento que ofrece en el uso de la cámara es de hasta 30 fps para los modos YUV420, RGB y en escala de grises con imágenes de hasta 1280x960 pixeles. En modo BGR y con un tamaño de 1280x960 el rendimiento cae hasta los 14 fps, ya que el proceso de conversión del espacio de color consume bastantes recursos. 

Por defecto, cuando se utiliza esta biblioteca con OpenCV las imágenes obtenidas estan en el espacio de color BGR, con lo que el desempeño es menor. Al utilizar una arquitectura cliente-servidor, se decide que la tarea de conversión sea delegada, y sea el servidor el encargado de cambiar al espacio de color de OpenCV cuando la imagen es recibida.

\subsection{Patrones utilizados}
El diseño de este subsistema corresponde con el patrón «factoría abstracta» [GHJV95], donde VideoCapture desempeña el papel de fábrica abstracta, VideoCaptureUEye y VideoCaptureOpenCV de fábricas concretas, y VideoFile y VideoDevice harían de productos.


La instanciación de un VideoCapture se realiza con una fachada (patrón facade), que se corresponde con la clase VideoCaptureFactory. Esta fachada sirve además para mantener una lista de objetos VideoCapture, que serán destruidos al finalizar la ejecución de forma transparente.


\section{Módulo de calibración}
Como ya se explicó en el capitulo \ref{chap:antecedentes}, los objetivos de realizar el proceso de calibración son la estimación de los parámetros intrínsecos y extrínsecos de la cámara. Los parámetros intrínsecos se refieren a las características internas de la cámara, como por ejemplo, su distancia focal, distorsión, y el centro de la imagen. Los parámetros extrínsecos describen su posición y orientación dentro de un espacio de refencia. Conocer los parámetros intrínsecos es un primer paso esencial, ya que permite calcular la estructura de la escena en el espacio euclideo y elimina la distorsión de lentes, la cual afecta a la precisión.

Se ha creado como una utilidad a parte del proceso principal. Ya que una vez calibrado el sistema, se proporcionan unos ficheros XML con los parametros intrinsecos y extrinsecos que se cargan en el proyecto. Mientras que la cámara y el proyector mantengan su posición y rotacion entre ellos, no es necesario realizar una nueva calibración y es posible mover todo el sistema.

El proceso de calibrado de la cámara esta basado esencialmente por el enfoque de [Zhang, 1999]. Se utiliza un patron tipo tablero de ajedrez, en la que se alternan cuadrados blancos y negros, de dimensiones conocidas. El patron se imprime y se pega sobre una superficie plana rígida.

A continuación se obtiene una serie de imágenes en los que se encuentre visible el patrón, con distintas orientaciones y distancias de la cámara. 

Se realiza el cálculo de las homografías entre el patrón y sus imágenes. Estas transformaciones proyectivas 2D producen un sistema de ecuaciones lineales que al resolverse obtiene los parámetros de la cámara. Esta fase generalmente es seguida por una etapa de refinamiento no lineal, basado en la minimización del error total de reproyección.

En principio, cualquier objeto caracterizado apropiadamente podría ser utilizado para la calibración. Existen otros métodos que basan sus referencias en objetos tridimensionales (por ejemplo, una caja cubierta con marcadores).  
La pricipal ventaja de la utilización de patrones planos, y que ha sido decisiva en la decisión del algoritmo a utilizar, es que son mucho más fáciles de tratar; resulta mucho más complicada la construcción y distribución de objetos 3D precisos para realizar una calibración.

El módulo implementado está basado en una extensión que han realizado Alvaro Cassinelli y Niklas Bergström a partir de un complemento de calibración desarrollado por Kyle McDonald que es capaz de calibrar cámaras y proyectores, consigiendo parametros intrínsecos de ambos, además de los extrinsecos en cuestión de varios minutos 

Aunque la cámara y el proyector podrían ser calibrados de forma simultánea, es mejor comenzar primero por calibrar la cámara.

El proyector proyecta un patron de circulos asimetrico, primero en una posición fija. La cámara se utiliza para calcular la posición 3D de los círculos proyectados, primero según el sistema de coordenadas de la camara, y luego segun el sistema de coordenadas del patron proyectado. Con esto parametros se calculan los parametros instrinsecos del proyector porque tiene puntos 3D (los círculos proyectados) en coordenadas reales, y sus respectivas proyecciones en el plano de imagen del proyector. El procedimiento de cálculo de homografiás es el mismo que para las cámaras, ya que el modelo matématico del proyector, es el de una cámara invertida. 

%(c) Finally, we can start computing the extrinsics of the camera-projector (because basically we have 3d points in "world coordinates" (the board), which are the projected circles, and also their projection (2d points) in the camera image and projector "image"). This means you can use the standard stereo calibration routine in openCV (this is done in the function. The reason why we compute FIRST the instrinsics of the projector is because the method stereoCalibrationCameraProjector will call the openCV stereo calibration function using "fixed intrinsics" to ensure better convergence of the algorithm (presumably better because if the camera is very well calibrated first, then the instrinsics of the projector should be good - probably better than if recomputed from all the 3d points and image points in camera and projector). 

%3) Once we get a good enough reprojection error (for the projector), we can start moving the projected points around so as to better explore the space (and get better and more accurate calibration). In this phase, we can run openCV stereo calibration to obtain the camera-projector extrinsics and again, we don't need to recompute the instrinics of the projector. After a few cycles (and cleaning of bad "boards"), the process converges, data is saved and you can do AR. 

%calibración de la cámara / proyector comienza: el proyector proyecta una rejilla de círculos (primero en una posición fija, a continuación, como la calibración logra cierta exactitud, la parrilla de salida siguiendo el patrón impreso). Este paso es el siguiente: (a) la cámara se utiliza para calcular la posición 3D de los círculos proyectados (por retroproyección) en la cámara de sistema de coordenadas, y luego en el "tablero" de coordenadas del sistema de coordenadas (o "mundo"). Esto se hace en el método de "backProject" (método de la calibración clase, pero llama sólo cuando el objeto es de tipo "cámara"). 

%Una vez calibrada, esta se se utiliza para calcular la posición 3D de los círculos proyectados (por retroproyección) en la cámara de sistema de coordenadas, y luego en el "tablero" de coordenadas del sistema de coordenadas (o "mundo"). Esto se hace en el método de "backProject" (método de la calibración clase, pero llama sólo cuando el objeto es de tipo "cámara"). 

%(b) Esto significa que podemos comenzar a computar los instrinsics del proyector porque tiene puntos 3d (los círculos proyectados) en coordenadas del mundo, y sus respectivos "proyección" "imagen" en el proyector plano. 

%(c) Finalmente, podemos empezar a calcular los extrinsics de la cámara-proyector (porque básicamente tenemos puntos 3d en "coordenadas mundo" (la mesa), que son los círculos proyectados, y también su proyección (puntos 2d) en la cámara imagen y proyector "imagen"). Esto significa que puede utilizar la rutina de calibración estéreo estándar en OpenCV (esto se hace en la función. La razón por la cual se calcula en primer lugar el instrinsics del proyector se debe a que el método stereoCalibrationCameraProjector llamará a la función de calibración OpenCV estéreo utilizando "intrínsecos fijos" para asegurar mejor convergencia del algoritmo (presumiblemente mejor, porque si la cámara está muy bien calibrado en primer lugar, a continuación, los instrinsics del proyector debe ser bueno - probablemente mejor que si recalculado desde todos los puntos 3D y puntos de imagen en la cámara y el proyector). 

%3) Una vez que tengamos una buena error reproyección suficiente (para el proyector), podemos empezar a mover los puntos proyectados en torno a fin de explorar mejor el espacio (y obtener una mejor y más precisa calibración). En esta fase, podemos ejecutar la calibración estéreo OpenCV para obtener los extrinsics cámara-proyector y otra vez, no necesitamos para volver a calcular los instrinics del proyector. Después de unos pocos ciclos (y limpieza de malas "tablas"), el proceso converge, los datos se guardan y se puede hacer AR. 



\subsection{Calibración de la cámara}

\subsection{Calibración del proyector}

%A continuación, la cámara se utiliza para calcular la posición 3D de los círculos proyectados (por retroproyección). Entonces podemos empezar a calibrar el proyector. Una vez que tengamos una buena error reproyección suficiente (para el proyector), podemos empezar a mover los puntos proyectados en torno a fin de explorar mejor el espacio (y obtener una mejor y más precisa calibración). En esta fase, podemos ejecutar la calibración estéreo OpenCV para obtener los parametros extrinsecos cámara-proyector. Después de unos pocos ciclos (y limpieza de malas "tablas"), el proceso converge, los datos se guardan y se puede hacer AR (*) 

%Básicamente, el procedimiento se divide en tres pasos: 

 %Notes: In case of camera/projector calibration, we need to proceed in TWO PHASES to give time to the projected image to refresh before trying to detect it (in case of "dynamic projected pattern"). Otherwise we may be detecting the OLD projected pattern, but using the newer image points (which completely breaks the calibration of course)
                
%PHASE 1 goal is just to set the projector image points to be projected, so the projector  will project something to be detected in the draw function. The first time, this is using the recorded pattern, but later (as the projector gets calibrated) we can use some arbitrary points "closer" to the printed pattern.

%In this later case (dynamic pattern), if the printed pattern is not visible, then we won't go to phase 2. 
                

%En caso de calibración de la cámara / proyector, tenemos que proceder en dos fases para dar tiempo a la imagen proyectada para refrescar antes de tratar de detectarlo (en caso de "patrón proyectado dinámica"). De lo contrario podemos estar detectando el viejo patrón proyectado, pero el uso de los puntos de imagen más recientes (que rompe por completo la calibración por supuesto) 

%FASE 1 gol es sólo para establecer los puntos de la imagen del proyector para ser proyectados, por lo que el proyector proyectará algo para ser detectado en la función draw. La primera vez, esto es usar el patrón grabado, pero más tarde (ya que el proyector se calibra) podemos utilizar algunos puntos arbitrarios "más cerca" del patrón impreso.



%El proceso de calibrado del proyector se divide en dos fases

%el proyector proyecta una rejilla de círculos (primero en una posición fija, a continuación, como la calibración logra cierta exactitud, la parrilla de salida siguiendo el patrón impreso). Este paso es el siguiente: 

%El funcionamiento del algoritmo de calibración se basa en encontrar las esquinas de los cuadrados en el patrón de calibración y realizar una asociación entre ellos. Además, habrá que proporcionar un valor de la medida real del mundo. Se proporciona la medida del lado de un cuadrado del tablero de ajedrez para ello y se utiliza un vector con la medida de las esquinas partiendo de esta base.

%Los puntos detectados en las imágenes se almacenan en un vector de objetos cv::Point2f. Para el programa se utilizará una matriz que contendrá dos vectores, uno por cada conjunto de puntos en una de las imágenes. Los puntos 3D que miden las distancias de las esquinas del tablero en unidades del mundo se almacenan en un vector de objetos cv::Point3f. Para detectar las esquinas en las imágenes se utiliza la función cv::findChessboardCorners(...). Adicionalmente, para mejorar esta detección se emplea cv::cornerSubPix, que se encargará de ubicar estas esquinas en medidas de subpíxels 1 .


%We use a pattern of alternating black and white squares (see Figure 11-9), which ensures that there is no bias toward one side or the other in measurement. Also, the resulting grid corners lend themselves naturally to the subpixel localization function discussed in Chapter 10. Given an image of a chessboard (or a person holding a chessboard, or any other scene with a chessboard and a reasonably uncluttered background), you can use the OpenCV function cvFindChessboardCorners() to locate the corners of the chessboard.

%In this routine, the method of calibration is to target the camera on a known structure that has many individual and identifi able points. By viewing this structure from a variety of angles, it is possible to then compute the (relative) location and orientation of the camera at the time of each  image as well as the intrinsic parameters of the camera (see Figure 11-9 in the “Chessboards” section). In order to provide multiple views, we rotate and translate the object, so let’s pause to learn a little more about rotation and translation.

%Imprimir un Select a pattern, download, and print Mount the pattern onto a rigid flat surface Take many pictures of the target at different orientations and distances Download pictures to compute and select ones that are in focus
\subsection{Cálculo de las matrices de transformación camara-proyector}


\section{Módulo de tracking y registro}
El cálculo del registro requiere posicionar el sistema cámara-proyector, mediante su posición y rotación, relativo a las hojas de papel que se encuentren en la escena capturada. Los métodos de tracking, en general, son los encargados de obtener una estimación de la trayectoria que realiza un objeto. En GrayAR, se ha optado por implementar un sistema de tracking visual mediante una aproximación \textit{bottom-up}\cite{Marimon}, en la que se cálculan los seis grados de libertad de la cámara (ver sección Pose) a partir de lo que se está percibiendo en la imagen. 

Vamos a aprovechar que conocemos la estructura de formato normalizado (según ISO 120/DIN 476) de una hoja de papel, para identificarlo, localizarlo en la escena capturada y establecer sus relaciones geometricas respecto a la cámara.

El enfoque utilizado es similar al de ARToolKit, mediante un algoritmo de detección de bordes y un método  de estimación de la orientación. La figura 3.7 resume el principio básico de funcionamiento de ARToolKit. Sobre la la imagen obtenida se inicia el primer paso de búsqueda de hojas de paperl. La imagen se convierte a blanco y negro para facilitar la detección de cuadrilateros; primero se convierte a escala de grises (b), y después se binariza (c) eligiendo un parámetro de umbral “threshold” que elige a partir de qué valor de gris (de entre 256 valores distintos) se considera blanco o negro (ver Figura 3.9). A continuación el algoritmo de visión por computador extrae componentes conectados de la imagen previamente binarizada, Figura 3.8 (d), cuya area es suficientemente grande como para detectar una marca. A estas regiones se les aplica un  algoritmo de detección de contornos (e), obteniendo a continuación los vértices y aristas que definen la región de la marca en 2D (f).

Conociendo las posiciones 2D de las aristas y vértices que definen el marcador 2D, y el modelo de proyección de la cámara es posible estimar la posición y rotación 3D de la cámara relativamente a la marca. El uso de marcas cuadradas de un tamaño previamente conocido nos permite definir un sistema de coordenadas local a cada marca, de modo que empleando métodos de visión por computador obtengamos la matriz de transformación 4x4 del sistema de coordenadas de la marca al sistema de coordenadas de la cámara.

\subsection{Conversión a escala de grises (Grayscale)}
El primer paso en el proceso de detección consiste en convertir el frame capturado por la cámara a escala de grises mediante la función \texttt{cv::cvtColor}. Esta función convierte una imagen de un espacio de color a otro. La constante \texttt{CV\_BGR2GRAY} sirve para indicar que queremos pasar de una imagen BGR (estandar de OpenCV) a otra en escala de grises.

\begin{listing}[
  float=ht,
  language = C++,
  caption  = {Conversión a escala de grises (256 niveles)},
  label    = code:cvtColor()]
  //it must be a 3 channel image
  if ( input.type() == CV_8UC3 )   
     cv::cvtColor ( input,grey,CV_BGR2GRAY );
  else 
     grey=input;
\end{listing}

\subsection{Umbralización o binarización (Thresholding)}
La umbralización consiste en definir un valor umbral y compararlo con cada uno de los pixels de la imagen. A los pixels que estén por debajo del umbral se les asigna un valor, y a los que estén por encima otro. De esta forma se divide toda la población de valores en tan sólo dos grupos, reduciendo considerablemente la complejidad de la información a analizar.

En GrayAR, se han desarrollado tres soluciones distintas para tratar de resolver este problema del thresholding básico (fijo, adaptativo y el método Canny). En un principio, solo se implementarion los metodos fijos y adaptativo, pero en una fase de optimzación realizada posteriormente, se incluyo el método Canny. Este algoritmo proporcionaba bordes mas precisos, y es mas torante a las variaciones de iluminación produciendo menores indices de ruido en la imagen binarizada. 

\begin{listing}[
  float=ht,
  language = C++,
  caption  = {Función PaperDetector::thresHold},
  label    = code:thresHold]
  void PaperDetector::thresHold(int method, const Mat &grey, Mat &out, double param1, double param2) throw (cv::Exception){
    if (param1 == -1) param1 = thresParam1;
    if (param2 == -1) param2 = thresParam2;
    
    if (grey.type() != CV_8UC1)     
      throw cv::Exception (9001,"grey.type() != CV_8UC1", "thresHold", __FILE__, __LINE__);
    
    switch (method){
    case FIXED_THRES:
      cv::threshold (grey, out, param1, 255, CV_THRESH_BINARY_INV);
      break;
    case ADPT_THRES://currently, this is the best method
      //ensure that _thresParam1%2==1
      if (param1 < 3) param1 = 3;
      else if (((int) param1) %2 != 1) param1 = (int)(param1 + 1);
      cv::adaptiveThreshold(grey, out, 255, ADAPTIVE_THRESH_MEAN_C, THRESH_BINARY_INV, param1, param2);
      break;
    case CANNY:
      cv::Canny (grey, out, param1, param2);
      break;
    }
  }
}
\end{listing}  
\subsection{Extracción de contornos (Suzuki)}
El siguiente paso del proceso es la detección de contornos, una operación un tanto distinta a las anteriores por dos motivos principales. El primero es que no consiste en iterar aplicando una misma función de forma monótona sobre todos los pixels de la imagen. Y el segundo es que el resultado que se obtiene no es una nueva imagen, sino un colección de conjuntos de puntos sobre la imagen.

Para este paso se parte de la imagen resultante del paso anterior. Es decir, de una imagen que tan solo consta de pixels con valor cero o uno. Sobre ella se aplica un algoritmo que la barre, empezando por su esquina superior izquierda, a la busca de un primer pixel a uno, y que cuando lo encuentra es capaz de seguir la cadena de pixels con valor uno que se encuentran unidos a él hasta volver al pixel de partida. Esa cadena de pixels a uno encontrados se denomina contorno. Y es más, el algoritmo es capaz de encontrar todos los contornos presentes en la imagen, ya que cuando termina con uno empieza de nuevo el proceso hasta asegurarse de haber barrido la imagen por completo.

El algoritmo es bastante potente, ya que no sólo es capaz de encontrar los contornos exteriores, sino también los interiores a otros, y retornarlos clasificados jerárquicamente. GrayAR utiliza la función \texttt{cv::findContours} que implementa este algoritmo. En la documentación, así como la referencia de implementación, se refiere al \textit{paper} original ``Topological structural analysis of digitized binary images by border following''  Satoshi Suzuki and Keiichi Abe.

\begin{listing}[]
cv::findContours ( thres2 , contours2, hierarchy2,CV_RETR_TREE, CV_CHAIN_APPROX_NONE );
\end{listing} 

Básicamente hay un contador de contornos encontrados y un buffer de pixels recorridos. El contador se inicializa a cero y el buffer con una copia de la imagen original. Se barre el buffer de arriba abajo y de izquierda a derecha. Una transición de un pixel 0 a otro 1 indica que se ha detectado un borde exterior, momento en que se suma uno al contador de contornos encontrados y se buscan todos los pixels 1 vecinos del encontrado. Si el pixel no tiene vecinos a 1 se cambia el valor del pixel por el del contador con signo contrario y se empieza otra vez con el siguiente pixel. En caso contrario se cambia el valor del pixel por el del contador, excepto que su valor sea mayor que 1, lo que significa que ya ha sido visitado, y se continúa buscando vecinos a 1 hasta retornar al pixel inicial. Una transición de un pixel con valor igual o mayor que 1 a otro 0 indica que se ha detectado un borde interior, momento en que se repite el mismo proceso usado para los contornos exteriores.

El algoritmo presenta una pequeña dificultad en los bordes de la imagen, y para solventarlo presupone que la imagen está rodeada por los cuatro lados de pixels con valor 0. Es decir, que el buffer que utiliza tiene una fila más por encima y por debajo, y una columna más a derecha e izquierda, que la imagen original.

\subsection{Aproximación a polígonos (Douglas-Peucker)}
Los contornos de los que se parte para este paso son colecciones de puntos. Todos y cada uno de los pixels que forman parte de ellos, ya que no se guardaron sólo las esquinas, sino cada pixel individual encontrado. Esto permite tomar algunas decisiones tempranas, como por ejemplo descartar los contornos que no tengan un mínimo de puntos. Es decir, los más pequeños. GrayAR descarta aquellos que no tengan al menos un 20\% del ancho de la imagen. 

GrayAR utiliza la función \texttt{cv::approxPolyDP}  para convertir una colección de puntos en un polígono mediante el algoritmo de Douglas-Peucker. El algoritmo admite como entrada una lista de puntos y una distancia máxima permitida. Define un segmento que va desde el primer punto de la lista hasta el último, y calcula la distancia más corta que hay desde dicho segmento a todos y cada uno del resto de puntos de la lista. Si encuentra un punto a una distancia mayor que la máxima pasada como parámetro divide la lista y el segmento en dos, utilizando el punto encontrado como nuevo extremo. Y vuelve a empezar el proceso comprobando cada nueva lista contra cada nuevo segmento de forma individual, creando nuevas listas y segmentos si fuera necesario, de forma recursiva.

\begin{listing}[]
approxPolyDP(contours2[i], approxCurve, double(contours2[i].size())*0.05, true);
\end{listing} 
 
GrayAR utiliza como distancia máxima para al algoritmo el 5\% del número de puntos que tiene cada contorno de forma individual. 

Una vez convertidos los puntos a polígonos se descartan los que no tengan cuatro lados, después selecciona sólo los polígonos convexos y también a aquellos que tengan un tamaño de lado mínimo. GrayAR concretamente descarta todos aquellos cuyo lado mínimo no supere los 10 pixels de largo. 

Aún realizando todos estos descartes, ArUco incorpora un control adicional, ya que debido a que en el paso previo se detectan contornos tanto exteriores como interiores, puede ocurrir que se detecten dos cuadriláteros, uno dentro de otro, siendo necesario descartar el más interno. GrayAR calcula la distancia media entre las esquinas de los polígonos y si encuentra dos que están muy cerca descarta el de menor perímetro, que necesariamente será el más interno. Entendiendo por ``muy cerca'' que se encuentren a menos de 10 pixels de distancia.

Por último, indicar que para el cálculo de las distancias entre esquinas es necesario un paso previo en el que las esquinas tienen que estar ordenadas en el sentido contrario de las agujas del reloj  \textit{(anti-clockwise order)}.

\subsection{Cálculo de parámetros extrínsecos de la cámara}
Una vez detectadas e identificadas las hojas de papel de la imagen, se procede a calcular los parámetros extrínsecos de la cámara respecto a cada una de ellas. Para la mayoría de los cálculos basta con considerar a la lente como un agujero puntual a través del que pasa la luz a través de un centro de proyección común. 

Digamos que vamos a visualizar un objeto que se encuentra a una distancia $x_{1}$ de la lente, y que la imagen proyectada por los rayos que salen de la lente se forma a una distancia $x_{2}$ tras la lente. Llamamos distancia focal a la relación que existe entre $x_{1}$ y $x_{2}$ :
\begin{center}
$\dfrac{1}{f} = \dfrac{1}{x_{1}} + \dfrac{1}{x_{2}}$
\end{center}

A la hora de representar los objetos del espacio observado (3D), los convertimos a un plano de visión (2D), y tenemos que definir cual es la correspondencia entre las coordenadas del espacio y las de la representación. Para facilitar esta conversión, se utiliza el sistema de coordenadas «homogéneo». Este sistema de coordenadas tiene la particularidad de que permite pasar fácilmente coordenadas de un número de dimensiones a otro. Para ello, almacena las coordenadas con una dimensión adicional, de tal forma que para un espacio de 3D, utilizaríamos 4 coordenadas. El valor de la coordenada adicional indica entre otras cosas, si el punto se encuentra en el infinito, valor 0, o es un punto cualquiera, valor distinto de 0. En este sistema, si dos coordenadas son proporcionales, se refieren al mismo punto.

Las matrices de transformación, utilizadas para pasar de unas dimensiones a otras, suelen incluir 3 tipos de transformaciones: Rotación, Translación y eScalado. Podemos representarlas así:
\begin{center}
$[R|t] = \begin{bmatrix} SR_{1,1} & SR_{1,2} & SR_{1,3} & T_{1}  \\
                                 SR_{2,1} & SR_{2,2} & SR_{2,3} & T_{2}  \\
                                 SR_{3,1} & SR_{3,2} & SR_{3,3} & T_{3}  \\
                                     0    &     0    &     0    &  1 
                  \end{bmatrix}
$
\end{center}

Para convertir las coordenadas del mundo a coordenadas de pantalla, de espacio tridimensional a bidimensional, los sistemas de visión artificial utilizan la siguiente fórmula:

\begin{center}
$q=MQ$ donde $q=\begin{bmatrix} x \\ y \\ z \end{bmatrix}$, $M=\begin{bmatrix} f_{x} & 0 & c_{x} \\ 0 & f_{y} & c_{y} \\ 0 & 0 & 1 \end{bmatrix}$ y $Q=\begin{bmatrix} X \\ Y \\ Z \end{bmatrix}$
\end{center}

Por tanto, $q$ representa el espacio bidimensional en coordenadas homogéneas, y $Q$ el espacio tridimensional. $f_{x}$ es la distancia focal en el eje $X$, $f_{y}$ la distancia focal en el eje $Y$, ($c_{x},c_{y}$) marcan el punto principal, el punto donde el eje de visión corta el plano de visión (normalmente suele estar en el centro de la imagen, o muy cerca) 

En GrayAR se uriliza la función de OpenCV \textit{solvePnp (ObjPoints, ImagePoints, camMatrix, distCoeff,raux,taux)}.   Está, calcula la pose (rotación y traslación) de la marca dado un conjunto de puntos de objeto(ObjPoints), sus proyecciones en la imagen correspondiente(ImagePoints), así como la matriz de cámara (camMatrix) y los coeficientes de distorsión (distCoeff). Esta función encuentra una pose que minimiza el error de reproyección, es decir, la suma de los cuadrados de las distancias entre las proyecciones imagePoints observados y los objectPoints proyectados

Se establece el centro del papel como centro de coordenadas (0,0,0), por lo que las esquinas que se le proporcionan a la función son :
\begin{center}
ObjPoints $=\begin{bmatrix} -h/2 & -h/2 & 0 \\
                 -h/2 &  h/2 & 0 \\
                  h/2 &  h/2 & 0 \\
                  h/2 & -h/2 & 0   
\end{bmatrix}$
Siendo \textbf{h} la longitud del lado de la marca
\end{center}
Los ImagePoints corresponden a los pixels que representan las esquinas de la marca detectada, camMatrix son los parametros intrinsecos de la cámara y distCoeff, el vector de coeficientes de distorsión. Estos ultimos parametros se obtienen  en un proceso previo de calibración de cámara y se mantienen fijos y conocidos durante todo el proceso.


%\begin{figure}
%  \centering
%    \includegraphics[width=0.7\textwidth]{marca}
%  \caption{Calculo de parametros extrinsecos}
%  \label{Figura 1}
%\end{figure}


\section{Módulo de detección de documentos}
\subsection{Transformación de perspectiva (Homography)}
Este paso admite como entrada la imagen en escala de grises creada en el primer paso, y los cuadriláteros candidatos encontrados en el último. Para cada cuadrilátero extrae la región de la imagen que cubre cada uno de ellos eliminando la transformación de perspectiva, es decir, la deformación que se produce en el marcador debido a la perspectiva.

GrayAR utiliza dos funciones de OpenCV para este paso: \texttt{cv::getPerspectiveTransform} y \texttt{cv::warpPerspective}. La primera calcula una matriz que multiplicada por un punto en el cuadrilátero origen devuelve un punto equivalente en el cuadrado destino. La segunda acepta una imagen, un cuadrilátero, una matriz de transformación, y retorna una nueva imagen del área cubierta por el cuadrilátero sobre la que se ha aplicado la matriz de transformación para eliminar la deformación debida a la perspectiva. 

El resultado de este proceso es cada una de las figuras encontradas en la imagen. Una para cada cuadrilátero candidato. GrayAR utiliza un tamaño de 56x56 pixels para los cuadrados destino.


\section{Módulo de interaccion natural}
\section{Módulo de soporte y utilidades}
\subsection{Gestor de configuración}
Existen muchos parametros configurables en GrayAR. En las primeras versiones del proyecto estos parametros estaban implementados en variables dentro del programa y cada vez que era necesario modificar estos parametros, por ejemplo para la realizacion de pruebas, era necesario recompilar el código. Para los archivos que se encuentran en la raspberry, el tiempo de compilación al modificar el valor de uno de esto parametros podia incluso a tardar varios minutos. Otro problema que surgió al crecer el proyecto esra que estos parametros estaban repartidos entre varios ficheros, lo que suponia tener que recordar donde estaba localizado cada parametro y el consecuente riesgo de dejarse alguno sin actualizar que invalidaria las pruebas realizadas.

Lo primero que se realizo fue sacar todos los parametros configurables a un fichero \acs{XML} que es leido al inicio del programa permitiendo que todos los parametros sean accesibles por cualquier módulo dentro del programa.

Las ventajas son apreciables a instante, el proceso de pruebas es mucho más rápido, ya que no es necesario volver a compilar cada vez que se realiza un cambio en la configuración. Todos los parametros se encuentran en un único fichero, que al ser \acs{XML}, tiene una estructura clara y legible para las personas, y que permite una edicion y modificacion sencilla. Tambien permite tener varios ficheros con distintas configuraciones y que se cargue en el programa uno u otro en funcion de las necesidades del momento.
 
La necesidad es la de tener una clase centralizada y accesible desde cualquier parte del programa que sea capaz de leer un fichero de configuración en \acs{XML} y almacene aquellas variables parametrizables de

Para implementar el gestor de configuración se ha creado la clase  \texttt{ConfigManager}. Esta tiene la funcion  \texttt{loadConfiguration} a la que se le indica el fichero \acs{XML} con la configuración a cargar 

Se ha optado por seguir el principio de diseño \acs{KISS} que establece que la mayoría de sistemas funcionan mejor si se mantienen simplesen lugar de hacerlos complejos; por ello, la simplicidad debe ser mantenida como un objetivo clave del diseño, y cualquier complejidad innecesaria debe ser evitada. Las ventajas genéricas que ofrece un patron Singleton frente a la implementación de un clase estática como puede ser que las funciones se puedan sobreescribir en las clases herederas, que la inicialización asincrona mientras que una clase estática generalmente se inicializa cuando se carga por primera vez o que el singleton pueda ser manejado polimorficamente sin forzar a los usuarios a asumir que solo existe una instancia, no suponen un mejora para la funcionalidad que necesitamos implementar. 

\subsection{Funciones de representación para depuración}
Aunque la creación de funciones para la representación gráfica están fuera del alcance de este proyecto, a la hora de probar y depurar los distintos algoritmos implementados resulta muy practico disponer de algunas funciones de dibujado ya implementadas para corregir errores y tener una vision preeliminar de como puede quedar finalmente.

A partir de las primitivas de dibujado que ofrece OpenCV se han implementado una serie de funciones para facilitar la representación de contornos, esquinas, y una vez obtenidos los parametros extrinsecos de las camara y el proyector, dibujado de ejes, cubos u ortoedros que permiten validar si los cálculos se han realizado correctamente.





