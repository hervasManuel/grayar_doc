\chapter{Resumen}
Las Interfaces Naturales de Usuario surgen como consecuencia de la evolución en la forma que tenemos para interaccionar con un computador. Su objetivo es lograr una tecnología lo más natural e intuitiva posible, de modo que el propio uso del ordenador pase desapercibido. 

El potencial de la Realidad Aumentada para construir interfaces naturales es evidente. Su uso va más allá de juegos y entretenimiento; las grandes compañías informáticas están presentando novedosos sistemas que explotan sus características. Aunque actualmente la mayoría de aplicaciones utilizan pantallas o gafas como dispositivos de representación, la utilización de proyectores permite desarrollar innovadoras \textit{«ventanas de información»} sobre cualquier superficie o dispositivo. 

La aplicación de técnicas de Realidad Aumentada en el ámbito de los Interfaces Naturales permite modelar entornos interactivos que proporcionen una sensación inmersiva al usuario sobre elementos existentes del entorno físico. 

El presente Trabajo Fin de Grado (con acrónimo GrayAR) surge como parte del \textit{Proyecto ARgos}, cuyo objetivo es el uso de técnicas de Realidad Aumentada como ayuda a la gestión documental. El proyecto plantea la construcción de un dispositivo embebido que dé soporte a la detección de documentos y su posterior tratamiento empleando técnicas de realidad aumentada mediante un entorno natural e interactivo.

GrayAR se encarga de la detección e identificación de documentos, así como del cálculo de su posicionamiento dentro del espacio 3D. El sistema emplea el paradigma de \textit{«pantalla táctil»}, utilizando el propio documento impreso como superficie interactiva. En GrayAR se ha elaborado una herramienta genérica que permite tanto el calibrado de cámaras y proyectores en el espacio 3D, como la representación de información aumentada directamente sobre el espacio físico. Además se ha construido un prototipo hardware sobre el que se han definido diferentes escenarios de explotación por la Asociación ASPRONA (Asociación para la Atención a Personas con Discapacidad Intelectual y sus Familias de la Provincia de Albacete).

\chapter{Abstract}

Natural User Interfaces arise from changes in the way that we have to interact with a computer. Its aim is to achieve the most natural and intuitive technology possible, so that the use of the computer itself goes unnoticed.

The potential of Augmented Reality to build natural interfaces is evident. Its use goes beyond gaming and entertainment; main computer companies are introducing new systems that exploit its features. Although most applications currently used screens or glass display devices, the use of projectors allows developing innovative «information windows» on any surface or device.

Application of Augmented Reality techniques in the field of Natural Interfaces allows modeling interactive environments that provide an immersive sensation to the user on existing elements of the physical environment.

This end-of-degree project (named GrayAR) comes as part of the \textit{ARgos Project} whose goal is the use of Augmented Reality techniques as an aid to document management. The project involves the construction of an embedded device that supports document image recognition and its management using augmented reality techniques through a natural and interactive environment.

GrayAR is responsible for the detection and identification of documents and calculating their position within the 3D space. The system uses the paradigm of «touch screen», using the printed document as an interactive surface. In GrayAR has developed a generic tool that allows calibration of cameras and projectors in 3D space, such as increased representation of information directly on the physical space. It has also built a prototype hardware on which they have defined different operational scenarios for ASPRONA Association (Association for the Care of Persons with Intellectual Disabilities and their Families of the Province of Albacete).