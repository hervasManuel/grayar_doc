\chapter{Resumen}

Las Interfaces Naturales de Usuario son la consecuencia de la evolución en la forma que tenemos para interaccionar con un computador. Su objetivo es lograr una tecnología natural e intuitiva y conseguir que nos pase desapercibido que estamos frente a un ordenador.

El potencial de la Realidad Aumentada para construir interfaces naturales es más que evidente. Su uso va más allá de juegos y entretenimiento; las grandes compañías informáticas están presentando novedosos sistemas de \textit{«computación humana»} que explotan sus características. Aunque actualmente, la mayoría de aplicaciones utilizan pantallas o gafas como dispositivos de representación, la utilización de proyectores permite desarrollar nuevas e innovadoras \textit{«ventanas de información»} que van más allá de las restricciones existentes en las pantalla planas.

La combinación de ambos enfoques, permite modelar entornos interactivos naturales, que proporcionen una sensación inmersiva al usuario y con la ventaja de que se pueden realizar sobre espacios cotidianos.

El presente Trabajo Fin de Grado surge como parte del \textit{Proyecto ARgos}, cuyo objetivo es el uso de técnicas de Realidad Aumentada como ayuda a la gestión documental en usuarios con necesidades especiales. En este proyecto se plantea la construcción de un dispositivo embebido que de soporte a la detección de documentos y su posterior tratamiento empleando técnicas de realidad aumentada mediante un entorno natural e interactivo.

Dentro del proyecto, GrayAR se encarga de la detección e identificación de documentos, así como del cálculo de su posicionamiento dentro del espacio 3D. Implementa un modelo de interacción natural con el usuario basado en el paradigma de \textit{«pantalla táctil»}, utilizando el propio documento impreso como superficie. Para lograrlo, se ha elaborado una herramienta genérica que permite el calibrado de cámaras y proyectores, y cuya finalidad, es la representación de información aumentada directamente sobre el espacio físico. Además se ha construido un prototipo hardware que muestra su funcionamiento en diferentes escenarios reales, definidos en el ámbito de la Asociación ASPRONA (Asociación para la Atención a Personas con Discapacidad Intelectual y sus Familias de la Provincia de Albacete).

\chapter{Abstract}

English version of the previous page.
