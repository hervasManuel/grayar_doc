\chapter{Manual de usuario}
\label{chap:anexo_manual}

El presente anexo constituye un manual de usuario, que abarca desde la instalación del sistema hasta
su uso.

\section{Instalación del cliente}

Suponiendo que contamos con una Raspberry Pi con la distribución Raspbian correctamente instalada,
la instalación de la aplicación cliente se divide en dos fases; instalación de dependencias y
compilación del sistema.

\subsection{Dependencias}

El sistema requiere las siguientes dependencias para su funcionamiento:

\vspace{-0.1cm}

\begin{listing}[%
  style    = consola]
libx11-dev
libfreetype6-dev
libsoil-dev
libsdl-1.2-dev
libboost-all-dev
libsdl-mixer-1.2-dev
libopencv-dev
\end{listing}

\vspace{-0.1cm}

Dichas dependencias pueden ser instaladas mediante el uso del comando \texttt{apt-get}:

\vspace{-0.1cm}

\begin{listing}[%
  style=consola]
# apt-get install libx11-dev libfreetype6-dev libsoil-dev \
    libsdl-1.2-dev libboost-all-dev libsdl-mixer-1.2-dev \
    libopencv-dev
\end{listing}

\vspace{-0.1cm}

Para utilizar la cámara CSI se requiere además la biblioteca
«RaspiCam»~\footnote{\url{http://www.uco.es/investiga/grupos/ava/node/40}}. Una vez descargada se procede
a su compilación e instalación:

\vspace{-0.1cm}

\begin{listing}[%
  style    = consola]
$ tar xvzf raspicamxx.tgz
$ cd raspicamxx
$ mkdir build
$ cd build
$ cmake ..
$ make
# make install
# ldconfig
\end{listing}

\subsection{Compilación}

Una vez instaladas las dependencias, se procede a compilar el sistema. Para ello, simplemente se
ejecuta desde el directorio raíz de la aplicación:

\begin{listing}[%
  style=consola]
$ make
\end{listing}

\section{Instalación del servidor}

La instalación del servidor sigue el mismo proceso de antes, solo que puede ser llevada a cabo en
cualquier computador.

\subsection{Dependencias}

El sistema requiere las siguientes dependencias para su funcionamiento:
\begin{listing}[%
  style=consola]
build-essential
cmake
git
libgtk2.0-dev
pkg-config
libavcodec-dev
libavformat-dev
libswscale-dev
python-dev
python-numpy
libtbb2
libtbb-dev
libjpeg-dev
libpng-dev
libtiff-dev
libjasper-dev
libdc1394-22-dev
\end{listing}

Dichas dependencias pueden ser instaladas mediante el uso del comando \texttt{apt-get}:

\begin{listing}[%
  style=consola]
# apt-get install build-essential cmake git libgtk2.0-dev \
    pkg-config libavcodec-dev libavformat-dev libswscale-dev \
    python-dev python-numpy libtbb2 libtbb-dev libjpeg-dev \
    libpng-dev libtiff-dev libjasper-dev libdc1394-22-dev
\end{listing}

Después, se pasa a compilar la biblioteca OpenCV:

\begin{listing}[%
  style=consola]
$ git clone https://github.com/Itseez/opencv.git
$ cd opencv
$ mkdir release
$ cd release
$ cmake -D CMAKE_BUILD_TYPE=RELEASE \
  -D CMAKE_INSTALL_PREFIX=/usr/local ..
$ make
# make install
\end{listing}

\subsection{Compilación}

Una vez instaladas las dependencias, se procede a compilar el sistema. Para ello, simplemente se
ejecuta desde el directorio raíz de la aplicación:

\begin{listing}[%
  style=consola]
$ make
\end{listing}

\section{Guía de uso}

En este apartado, se introduce el funcionamiento del sistema de tal forma que sirva de guía al
usuario.

\subsection{Instalación de recursos}

Los recursos como imágenes, sonidos, vídeos y fuentes, deben de respetar las rutas de directorios
existentes:

\begin{itemize}
\item Los archivos de imagen deben de ir en el directorio \texttt{media/images} del cliente.
\item Los archivos de sonido deben de ir en el directorio \texttt{media/sounds} del cliente.
\item Los archivos de vídeo deben de ir en el directorio \texttt{media/videos} del cliente.
\item Los archivos de fuentes \textit{true-type} deben de ir en el directorio \texttt{media/fonts}
  del cliente.
\item Los archivos de \textit{script} ARS y configuración XML deben de ir en el
  directorio \texttt{data} del servidor.
\end{itemize}

\subsection{Calibración del sistema}

El proceso de calibrado del sistema se ha creado como una utilidad independiente del sistema
principal de ARgos. Una vez calibrado el sistema, la aplicación proporciona los ficheros XML, con
los parámetros intrínsecos y extrínsecos, que necesita ARgos para el calculo del registro.

El proceso de calibrado de la cámara esta basado esencialmente por el enfoque de Zhang
~\cite{Zhang:1999:FCCVPFUO}. Se utiliza un patrón tipo tablero de ajedrez, en la que se alternan
cuadrados blancos y negros, de dimensiones conocidas. El patrón se imprime y se pega sobre una
superficie plana rígida. Se debe dejar una zona despejada a continuación del tablero de ajedrez, ya
que en esta zona se proyectará el patrón que utiliza el proyector para su calibrado.

A continuación, colocamos la superficie bajo la cámara, en distintas orientaciones y distancias,
para obtener una serie de imágenes en los que se encuentre visible el patrón.

Cuando el proceso disponga de al menos 20 imágenes válidas, realizará los cálculos para obtener los
parámetros intrínsecos de cámara y los grabará en el fichero \textbf{calibrationCamera.xml}

El siguiente paso consiste en calibrar el proyector. Se proyecta un patrón de círculos asimétrico,
en una posición fija. Colocar la superficie de tal forma que el patrón proyectado quede junto al del
tablero de ajedrez. Volver a mover el patrón por espacio de proyección hasta que el sistema adquiera
5 imágenes. En ese momento, la calibración pasa al modo dinámico, y los puntos proyectados se
generan en función de la posición que tenga tablero de ajedrez.

Tras capturar 15 imágenes con los puntos dinámicos, se realizan los cálculos de los parámetros
intrínsecos del proyector, y finalmente, de los parámetros extrínsecos (matrices de rotación y
traslación) entre la cámara y el proyector.

Los parámetros de calibrado se almacenan en los ficheros \textbf{calibrationProjector.xml} y
\textbf{cameraProjectorExtrinsics.xml}.

Mientras que la cámara y el proyector mantengan su posición y rotación entre ellos, no es necesario
realizar una nueva calibración y es posible mover todo el sistema.

Una vez realizada la calibración se iniciará un test para comprobar que se ha hecho
correctamente. El test consiste en proyectar un círculo sobre las cuatro esquinas exteriores del
patrón de tablero de ajedrez. Al mover el patrón los puntos proyectados deben siempre permanecer
alineados con las esquinas.

\subsection{Descripción de documentos}

La descripción de los documentos que se puedan identificar en el sistema se realiza proporcionando
una imagen en formato JPEG de una plantilla del documento.

\subsection{Fichero de configuración}

El archivo de configuración permite cambiar algunos parámetros del sistema. El archivo se encuentra
en la carpeta \texttt{data} del servidor y presenta una estructura tal que:

\begin{listing}[%
  language = XML]
<config>
  <paper_size width="21.0" height="29.7" />
  <calibration_files camera="calibrationCamera.xml"
    projector="calibrationProjector.xml"
    extrinsics="CameraProjectorExtrinsics.xml" />
  <argos_papers>
    <paper id="0" description="Neobiz"
      scriptFileName="NeobizScript.ars"
      descriptorFileName="neobiz-.jpg" />
    <paper id="1" description="Sinovo"
      scriptFileName="SinovoScript.ars"
      descriptorFileName="sinovo-.jpg" />
    <paper id="2" description="Active"
      scriptFileName="ActiveScript.ars"
      descriptorFileName="active-.jpg" />
  </argos_papers>
  <output_display type="projector" />
</config>
\end{listing}

El elemento \texttt{paper\_size} indica las dimensiones de los documentos sobre los que se va a
trabajar. En este caso, representa el tamaño de un papel en formato A4, es decir, 21.0x29.7 cm.

El elemento \texttt{calibration\_files} indica los nombres de los archivos de calibración generados
en el apartado anterior.

El elemento \texttt{argos\_papers} define la lista de documentos con los que va a trabajar el
sistema. Estos documentos mantienen una asociación con un archivo de \textit{script} ARS y su
imagen correspondiente.

El elemento \texttt{output\_display} indica al sistema si la visualización la va a realizar un
proyector o un monitor para realizar los cálculos del registro en función del dispositivo.

\subsection{Creación de \textit{scripts}}

Para la creación de \textit{scripts} se debe crear un nuevo documento de texto plano con cualquier
editor de texto soportado y escribir en él las sentencias que deseemos.

\subsubsection{Sintaxis}

Las sentencias siguen el formato:

%\lstinline{<carácter_de_control> <función> <arg_1> <arg_2> <arg_n> ...}

Donde cada elemento de la sentencia significa:
\begin{itemize}
\item \textbf{<caracter de control>:} Puede ser uno de los siguientes:
  \begin{itemize}
  \item \texttt{\#} Indica que esta sentencia será ignorada (no se ejecutará).
  \item \texttt{>} La sentencia precedida por este carácter será ejecutada.
  \item \texttt{!} Una vez encontrado este carácter, el resto del \textit{script} será ignorado
    (no se ejecutará).
  \end{itemize}
\item \textbf{<función>:} Se trata del nombre de la función a invocar.
\item \textbf{<arg\_1> <arg\_2> <arg\_n> ...:} Los argumentos de la función anterior.
\end{itemize}

\subsubsection{Funciones}

A continuación se especifica la lista de funciones disponibles a invocar desde un \textit{script}.

\begin{description}
\item[DrawImage] \hfill \\
  \textsc{Descripción:} Dibuja una imagen desde el disco. \\
  \textsc{Argumentos:}
  \begin{itemize}
  \item filename: \textit{string}. El nombre del archivo de imagen a dibujar.
  \item x: \textit{float}. La coordenada X donde situar la imagen.
  \item y: \textit{float}. La coordenada Y donde situar la imagen.
  \item z: \textit{float}. La coordenada Z donde situar la imagen.
  \item w: \textit{float}. El ancho de la imagen (1.0 = 1:1).
  \item h: \textit{float}. El alto de la imagen (1.0 = 1:1).
  \end{itemize}
\item[DrawVideo] \hfill \\
  \textsc{Descripción:} Dibuja un vídeo desde el disco. \\
  \textsc{Argumentos:}
  \begin{itemize}
  \item filename: \textit{string}. El nombre del archivo de video a dibujar.
  \item x: \textit{float}. La coordenada X donde situar la video.
  \item y: \textit{float}. La coordenada Y donde situar la video.
  \item z: \textit{float}. La coordenada Z donde situar la video.
  \item w: \textit{float}. El ancho de la imagen (1.0 = 1:1).
  \item h: \textit{float}. El alto de la imagen (1.0 = 1:1).
  \end{itemize}
\item[DrawCorners] \hfill \\
  \textsc{Descripción:} Dibuja esquinas sobre el documento. \\
  \textsc{Argumentos:}
  \begin{itemize}
  \item length: \textit{float}. La longitud de segmento de cada esquina.
  \item wide: \textit{float}. El ancho de segmento de cada esquina.
  \item r: \textit{float}. La componente de color roja para las esquinas, entre 0.0 y 1.0.
  \item g: \textit{float}. La componente de color verde para las esquinas, entre 0.0 y 1.0.
  \item b: \textit{float}. La componente de color azul para las esquinas, entre 0.0 y 1.0.
  \item w: \textit{float}. El ancho del documento. Un A4 sería 21.
  \item h: \textit{float}. El alto del documento. Un A4 sería 29.7.
  \end{itemize}
\item[DrawAxis] \hfill \\
  \textsc{Descripción:} Dibuja un eje de coordenadas. \\
  \textsc{Argumentos:}
  \begin{itemize}
  \item length: \textit{float}. La longitud de segmento de cada eje.
  \item wide: \textit{float}. El ancho de segmento de cada eje.
  \item x: \textit{float}. La coordenada X donde situar el eje de coordenadas.
  \item y: \textit{float}. La coordenada Y donde situar el eje de coordenadas.
  \item z: \textit{float}. La coordenada Z donde situar el eje de coordenadas.
  \end{itemize}
\item[InitVideostream] \hfill \\
  \textsc{Descripción:} Inicia una videollamada. \\
  \textsc{Argumentos:}
  \begin{itemize}
  \item bg\_file: \textit{string}. El nombre del archivo de imagen a cargar de fondo para la videollamada.
  \item w: \textit{float}. El ancho de la videollamada.
  \item h: \textit{float}. El alto de la videollamada.
  \item port: \textit{int}. El número de puerto donde iniciar la videollamada.
  \end{itemize}
\item[DrawTextPanel] \hfill \\
  \textsc{Descripción:} Dibuja un panel de texto. \\
  \textsc{Argumentos:}
  \begin{itemize}
  \item r: \textit{float}. La componente de color roja de este \texttt{TextPanel}, entre 0.0 y 1.0.
  \item g: \textit{float}. La componente de color verde de este \texttt{TextPanel}, entre 0.0 y 1.0.
  \item b: \textit{float}. La componente de color azul de este \texttt{TextPanel}, entre 0.0 y 1.0.
  \item text: \textit{string}. El texto que contendrá este \texttt{TextPanel}.
  \item x: \textit{float}. La coordenada X donde situar el \texttt{TextPanel}.
  \item y: \textit{float}. La coordenada Y donde situar el \texttt{TextPanel}.
  \item z: \textit{float}. La coordenada Z donde situar el \texttt{TextPanel}.
  \end{itemize}
\item[DrawHighlight] \hfill \\
  \textsc{Descripción:} Dibuja un área de resaltado. \\
  \textsc{Argumentos:}
  \begin{itemize}
  \item r: \textit{float}. La componente de color roja para este \texttt{Highlight}, entre 0.0 y 1.0.
  \item g: \textit{float}. La componente de color verde para este \texttt{Highlight}, entre 0.0 y 1.0.
  \item b: \textit{float}. La componente de color azul para este \texttt{Highlight}, entre 0.0 y 1.0.
  \item x: \textit{float}. La coordenada X donde situar este \texttt{Highlight}.
  \item y: \textit{float}. La coordenada Y donde situar este \texttt{Highlight}.
  \item z: \textit{float}. La coordenada Z donde situar este \texttt{Highlight}.
  \item w: \textit{float}. El ancho de este \texttt{Highlight} (1.0 = 1:1).
  \item h: \textit{float}. El alto de este \texttt{Highlight} (1.0 = 1:1).
  \end{itemize}
\item[DrawButton] \hfill \\
  \textsc{Descripción:} Dibuja un botón. \\
  \textsc{Argumentos:}
  \begin{itemize}
  \item r: \textit{float}. La componente de color roja para este botón, entre 0.0 y 1.0.
  \item g: \textit{float}. La componente de color verde para este botón, entre 0.0 y 1.0.
  \item b: \textit{float}. La componente de color azul para este botón, entre 0.0 y
    1.0.
  \item text: \textit{string}. El texto de este botón.
  \item x: \textit{float}. La coordenada X donde situar este botón.
  \item y: \textit{float}. La coordenada Y donde situar este botón.
  \item z: \textit{float}. La coordenada Z donde situar este botón.
  \end{itemize}
\item[DrawFactureHint] \hfill \\
  \textsc{Descripción:} Dibuja un cuadro con instrucciones para tratar el documento. \\
  \textsc{Argumentos:}
  \begin{itemize}
  \item x: \textit{float}. La coordenada X donde situar esta ayuda.
  \item y: \textit{float}. La coordenada Y donde situar esta ayuda.
  \item z: \textit{float}. La coordenada Z donde situar esta ayuda.
  \item w: \textit{float}. El ancho de esta ayuda (1.0 = 1:1).
  \item h: \textit{float}. El alto de esta ayuda (1.0 = 1:1).
  \item r: \textit{float}. La componente de color roja para esta ayuda, entre 0.0 y 1.0.
  \item g: \textit{float}. La componente de color verde para esta ayuda, entre 0.0 y 1.0.
  \item b: \textit{float}. La componente de color azul para esta ayuda, entre 0.0 y
    1.0.
  \item title: \textit{string}. El texto de título de esta ayuda.
  \item text\_block\_1: \textit{string}. El primer bloque de texto de esta ayuda.
  \item text\_block\_2: \textit{string}. El segundo bloque de texto de esta ayuda.
  \end{itemize}
\item[PlaySound] \hfill \\
  \textsc{Descripción:} Reproduce un sonido. \\
  \textsc{Argumentos:}
  \begin{itemize}
  \item filename: \textit{string}. El nombre del archivo de sonido a reproducir.
  \item loops: \textit{int}. El número de veces a reproducir el sonido. -1 indefinidamente, 0
    reproduce una vez, 1 reproduce dos veces, etc.
  \end{itemize}
\item[PlaySoundDelayed] \hfill \\
  \textsc{Descripción:} Reproduce un sonido en bucle con un retraso de tiempo entre reproducciones indicado. \\
  \textsc{Argumentos:}
  \begin{itemize}
  \item filename: \textit{string}. El nombre del archivo de sonido a reproducir.
  \item delay: \textit{float}. El número de segundos a esperar entre reproducciones.
  \end{itemize}
\item[CheckRegion] \hfill \\
  \textsc{Descripción:} Comprueba el documento dentro de la región rectangular indicada. \\
  \textsc{Argumentos:}
  \begin{itemize}
  \item x: \textit{float}. La posición X de la región a comprobar (esquina superior izquierda).
  \item y: \textit{float}. La posición Y de la región a comprobar (esquina superior izquierda).
  \item w: \textit{float}. El ancho de la región a comprobar (esquina inferior derecha).
  \item h: \textit{float}. El alto Y de la región a comprobar (esquina inferior derecha).
  \end{itemize}
\end{description}

\subsection{Ejecución}

El arranque del sistema requiere ejecutar de forma separada tanto el servidor como el cliente, y en
ese orden para obtener la dirección de conexión.

Ejecución del servidor. Se debe indicar el número de puerto y la interfaz de red donde escuchará el
servidor:

\begin{listing}[%
  style=consola]
$ ./argos_server <port> <iface>}
\end{listing}

Ejecución del cliente. Se debe indicar la dirección IP y el puerto proporcionados por el
servidor:

\begin{listing}[%
  style=consola]
$ ./argos_client <ip>:<port>
\end{listing}

Una vez ejecutado solo queda interactuar en la superficie de trabajo con los documentos definidos
en los descriptores.

% Local Variables:
% TeX-master: "main.tex"
%  coding: utf-8
%  mode: latex
%  mode: flyspell
%  ispell-local-dictionary: "castellano8"
% End:
