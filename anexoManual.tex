\chapter{Manual de usuario}
\label{chap:anexo_manual}

El presente anexo constituye un manual de usuario, que abarca desde la instalación del sistema hasta
su uso.

\section{Instalación del cliente}

Suponiendo que contamos con una Raspberry Pi con la distribución Raspbian correctamente instalada,
la instalación de la aplicación cliente se divide en dos fases; instalación de dependencias y
compilación del sistema.

\subsection{Dependencias}

El sistema requiere las siguientes dependencias para su funcionamiento:

\vspace{-0.1cm}

\begin{listing}[%
  style    = consola]
libx11-dev
libfreetype6-dev
libsoil-dev
libsdl-1.2-dev
libboost-all-dev
libsdl-mixer-1.2-dev
libopencv-dev
\end{listing}

\vspace{-0.1cm}

Dichas dependencias pueden ser instaladas mediante el uso del comando \texttt{apt-get}:

\vspace{-0.1cm}

\begin{listing}[%
  style=consola]
# apt-get install libx11-dev libfreetype6-dev libsoil-dev \
    libsdl-1.2-dev libboost-all-dev libsdl-mixer-1.2-dev \
    libopencv-dev
\end{listing}

\vspace{-0.1cm}

Para utilizar la cámara CSI se requiere además la biblioteca
«RaspiCam»~\footnote{\url{http://www.uco.es/investiga/grupos/ava/node/40}}. Una vez descargada se procede
a su compilación e instalación:

\vspace{-0.1cm}

\begin{listing}[%
  style    = consola]
$ tar xvzf raspicamxx.tgz
$ cd raspicamxx
$ mkdir build
$ cd build
$ cmake ..
$ make
# make install
# ldconfig
\end{listing}

\subsection{Compilación}

Una vez instaladas las dependencias, se procede a compilar el sistema. Para ello, simplemente se
ejecuta desde el directorio raíz de la aplicación:

\begin{listing}[%
  style=consola]
$ make
\end{listing}

\section{Instalación del servidor}

La instalación del servidor sigue el mismo proceso de antes, solo que puede ser llevada a cabo en
cualquier computador.

\subsection{Dependencias}

El sistema requiere las siguientes dependencias para su funcionamiento:
\begin{listing}[%
  style=consola]
build-essential
cmake
git
libgtk2.0-dev
pkg-config
libavcodec-dev
libavformat-dev
libswscale-dev
python-dev
python-numpy
libtbb2
libtbb-dev
libjpeg-dev
libpng-dev
libtiff-dev
libjasper-dev
libdc1394-22-dev
\end{listing}

Dichas dependencias pueden ser instaladas mediante el uso del comando \texttt{apt-get}:

\begin{listing}[%
  style=consola]
# apt-get install build-essential cmake git libgtk2.0-dev \
    pkg-config libavcodec-dev libavformat-dev libswscale-dev \
    python-dev python-numpy libtbb2 libtbb-dev libjpeg-dev \
    libpng-dev libtiff-dev libjasper-dev libdc1394-22-dev
\end{listing}

Después, se pasa a compilar la biblioteca OpenCV:

\begin{listing}[%
  style=consola]
$ git clone https://github.com/Itseez/opencv.git
$ cd opencv
$ mkdir release
$ cd release
$ cmake -D CMAKE_BUILD_TYPE=RELEASE \
  -D CMAKE_INSTALL_PREFIX=/usr/local ..
$ make
# make install
\end{listing}

\subsection{Compilación}

Una vez instaladas las dependencias, se procede a compilar el sistema. Para ello, simplemente se
ejecuta desde el directorio raíz de la aplicación:

\begin{listing}[%
  style=consola]
$ make
\end{listing}

\section{Guía de uso}

En este apartado, se introduce el funcionamiento del sistema de tal forma que sirva de guía al
usuario.

\subsection{Instalación de recursos}

Los recursos como imágenes, sonidos, vídeos y fuentes, deben de respetar las rutas de directorios
existentes:

\begin{itemize}
\item Los archivos de imagen deben de ir en el directorio \texttt{media/images} del cliente.
\item Los archivos de sonido deben de ir en el directorio \texttt{media/sounds} del cliente.
\item Los archivos de vídeo deben de ir en el directorio \texttt{media/videos} del cliente.
\item Los archivos de fuentes \textit{true-type} deben de ir en el directorio \texttt{media/fonts}
  del cliente.
\item Los archivos de \textit{script} ARS y configuración XML deben de ir en el
  directorio \texttt{data} del servidor.
\end{itemize}

\subsection{Calibración del sistema}

El proceso de calibrado del sistema se ha creado como una utilidad independiente del sistema
principal de ARgos. Una vez calibrado el sistema, la aplicación proporciona los ficheros XML, con
los parámetros intrínsecos y extrínsecos, que necesita ARgos para el calculo del registro.

\subsection{Descripción de documentos}

La descripción de los documentos que se puedan identificar en el sistema se realiza proporcionando
una imagen en formato JPEG de una plantilla del documento.

\subsection{Fichero de configuración}

El archivo de configuración permite cambiar algunos parámetros del sistema. El archivo se encuentra
en la carpeta \texttt{data} del servidor y presenta una estructura tal que:

\begin{listing}[%
  language = XML]
<config>
  <paper_size width="21.0" height="29.7" />
  <calibration_files camera="calibrationCamera.xml"
    projector="calibrationProjector.xml"
    extrinsics="CameraProjectorExtrinsics.xml" />
  <argos_papers>
    <paper id="0" description="Neobiz"
      scriptFileName="NeobizScript.ars"
      descriptorFileName="neobiz-.jpg" />
    <paper id="1" description="Sinovo"
      scriptFileName="SinovoScript.ars"
      descriptorFileName="sinovo-.jpg" />
    <paper id="2" description="Active"
      scriptFileName="ActiveScript.ars"
      descriptorFileName="active-.jpg" />
  </argos_papers>
  <output_display type="projector" />
</config>
\end{listing}

El elemento \texttt{paper\_size} indica las dimensiones de los documentos sobre los que se va a
trabajar. En este caso, representa el tamaño de un papel en formato A4, es decir, 21.0x29.7 cm.

El elemento \texttt{calibration\_files} indica los nombres de los archivos de calibración generados
en el apartado anterior.

El elemento \texttt{argos\_papers} define la lista de documentos con los que va a trabajar el
sistema. Estos documentos mantienen una asociación con un archivo de \textit{script} ARS y su
imagen correspondiente.

El elemento \texttt{output\_display} indica al sistema si la visualización la va a realizar un
proyector o un monitor para realizar los cálculos del registro en función del dispositivo.


\subsection{Ejecución}

El arranque del sistema requiere ejecutar de forma separada tanto el servidor como el cliente, y en
ese orden para obtener la dirección de conexión.

Ejecución del servidor. Se debe indicar el número de puerto y la interfaz de red donde escuchará el
servidor:

\begin{listing}[%
  style=consola]
$ ./argos_server <port> <iface>}
\end{listing}

Ejecución del cliente. Se debe indicar la dirección IP y el puerto proporcionados por el
servidor:

\begin{listing}[%
  style=consola]
$ ./argos_client <ip>:<port>
\end{listing}

Una vez ejecutado solo queda interactuar en la superficie de trabajo con los documentos definidos
en los descriptores.

% Local Variables:
% TeX-master: "main.tex"
%  coding: utf-8
%  mode: latex
%  mode: flyspell
%  ispell-local-dictionary: "castellano8"
% End:
